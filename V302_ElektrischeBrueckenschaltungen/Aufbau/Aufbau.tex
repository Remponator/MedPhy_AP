\newpage
\section{Aufbau und Durchführung}
Der Aufbau des Versuchs lässt sich prinzipiell durch die in Abbildung 1 skzzierte Brückenschaltung beschreiben.
Die Schaltung besteht grundlegend aus einer Parallelschaltung, deren beide Abzweigungen aus zwei in Reihe geschalteten Bauteilen, beispielsweise Widerständen,Induktivitäten oder Kapazitäten, besteht.
Mit einem Oszilloskop, vor dessen Eingang ein Tiefpass angeschlossen wird um hochfrequente Störspannungen zu vermeiden, wird die Brückenspannung $U_\text{Br}$ zwischen den Punkten A und B gemessen.
Als Spannungsquelle wird ein Sinusgenerator mit der Spannung $U_S$ eingesetzt.
Im Folgenden werden 5 verschiedene Schaltungen beschrieben, die im Experiment verwendet werden.

\renewcommand{\labelenumi}{\alph{enumi})}
\begin{enumerate}
\item \textbf{Wheatstonesche Brücke} \\
Die Wheatstonesche Brücke wird auch als Widerstandsbrücke bezeichnet und besteht, wie der Name schon sagt, nur aus ohmschen Widerständen.
Hiermit wird der Wert des Widerstands $R_X$ gemessen, indem das Potentiometer so eingestellt wird, dass die Brückenspannung $U_\text{Br}$ verschwindet.
\item  \textbf{Kapazitätsmessbrücke} \\
Die Kapazitätsmessbrücke besteht, wie die Abbildung 3 zeigt, zusätzlich aus Kondensatoren, die zu den Widerständen in Reihe geschaltet werden.
Hiermit lassen sich die Werte des Widerstands $R_X$ und der Kapazität $C_X$ realer Kondensatoren bestimmen.
Zur Berechnung dieser Werte wird das Verhältnis der Widerstände $R_2$, $R_3$ und $R_4$ so eingestellt, dass keine Brückenspannung $U_\text{Br}$ mehr zu messen ist.
\item \textbf{Induktivitätsmessbrücke} \\
Die Induktivitätsmessbrücke wird anhand Abbildung 4 analog zur Kapazitätsbrücke aufgebaut und durchgeführt.
Allerdings werden Spulen anstatt Kondensatoren verwendet und Induktivitäten anstatt Kapazitäten bestimmt.
\item \textbf{Maxwell-Brücke} \\
Bei der in Abbildung 5 dargestellten Maxwell-Brücke wird neben der Induktivitätsmessbrücke eine weitere Schaltung zur Bestimmung des Widerstandes und der Induktivität einer realen Spule aufgezeigt.
Hier werden erneut das Verhältnis der Widerstände $R_3$ und $R_4$ so eingestellt, dass keine Brückenspannung $U_\text{Br}$ mehr am Oszilloskop abgelesen werden kann.
\item \textbf{Wien-Robinson-Brücke} \\
In Abbildung 6 wird die Wien-Robinson-Brücke gezeigt, mit der man die Frequenzabhängigkeit eines Schwingkreises bestimmen kann.
Hierzu werden Kapazitäten und bestimmte, feste Verhältnisse von Widerständen und eine Kapazität verwendet.
Dabei wird die Wechselspannung $U_S$ auf verschiedene Frequenzen gestellt und die Brückenspannung $U_\text{Br}$ in Abhängigkeit der Frequenz gemessen.
\end{enumerate}
