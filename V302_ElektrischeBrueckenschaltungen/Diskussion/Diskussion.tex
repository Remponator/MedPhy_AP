\section{Diskussion}

In dem Versuch werden viele Fehler durch die Bauteile bestimmt. Zudem müssen Messungenauigkeiten der Messgeräte und Ungenauigkeiten beim Ablesen berücksichtigt werden.
Auffällig ist, dass die in 4.3 durch die Induktivitätsmessbrücke gemessene Induktivität einen kleineren Fehler ($\pm 0.144 \text{mH}$) als die bei 4.4 durch die Maxwell-Brücke gemessene Induktivität ($\pm 0.831 \text{mH}$) hat. Die Maxwell-Brücke ist durch die zusätzlich Kapazität mit mehr Fehlern behaftet als die Induktivitätsmessbrücke. Zudem ist der Innenwiderstand $R_{19}$ der Maxwell-Brücke signifikant größer.
\\
Anhand des Graphen \ref{fig:Frequenz} lässt sich erkennen, dass die Messungen mit dem Wien-Robinson-Filter sehr genau sind.  
Der ausgerechnete Klirrfaktor des Sinusgenerators (3,45$\%$) liegt im Rahmen der theoretischen Werte \cite{2}, wobei der Klirrfaktor jedoch standardmäßig im Bereich ziwschen 0,1$\%$ und 1$\%$ liegt.