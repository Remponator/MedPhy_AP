\section{Aufbau und Durchführung}
\label{sec:Aufbau}
\subsection{Ablenkung im E-Feld}

Der Aufbau der Kathodenstrahlröhre wurde in Abschnitt \ref{sec:Theorie} bereits behandelt.
Zur Messung der Ablenkung im E-Feld wird nun die Schaltung aus Abbildung \ref{fig:ESchaltung1} aufgebaut.
\begin{figure}
    \centering
    \includegraphics[height=4cm]{Aufbau/ESchaltung1.pdf}
    \caption{Schaltung zur Messung der Ablenkung durch ein E-Feld \cite{V501}.}
    \label{fig:ESchaltung1}
\end{figure}
Es wird die Proportionalität zwischen Verschiebung und Ablenkspannung $U_\text{d}$ für 5 verschiedene Beschleunigungsspannungen $U_\text{B}$ gemessen. $U_\text{B}$ liegt dabei zwischen 180 und 500V.
$U_\text{d}$ wird jedes Mal so eingestellt, dass der Leuchtpunkt auf den 9 äquidistanten Linien des Koordinatenetzes liegt. $U_\text{d}$ wird am Voltmeter abgelesen.
\\
Nun wird der Kathodenstrahl-Oszillosgraph nach Abbildung \ref{fig:ESchaltung2} aufgebaut.
\begin{figure}
    \centering
    \includegraphics[height=4cm]{Aufbau/ESchaltung2.pdf}
    \caption{Schaltung des Kathodenstrahl-Oszillographen \cite{V501}.}
    \label{fig:ESchaltung2}
\end{figure}
Durch Variation der Sägezahnspannung wird versucht ein das Bild einer stehenden Welle der Sinusspannung zu erreichen.
Hierzu werden die Verhältnisse $\text{n} v_\text{sä} = v_\text{si}$ mit n = 0,5 , 1, 2, 3. Dabei wird die maximale Strahlauslenkung durch die Sinusspanung in y-Richtung ausgemessen.

\subsection{Ablenkung im B-Feld}
 
 Mithilfe einer Helmholtzspule wird ein homogenes Magnetfeld erzeugt. Die Richtung des Magnetfeldes ist senkrecht zum Elektronenstrahl der Kahodenstrahlröhre. \\
 Vor BEginn der Messung wird die Achse des Kathodenstrahlrohres parallel zur Horizontalkomponente des Erdmagnetfeldes gerichtet. Hierbei hilft ein Deklinatorium-Inklinatorium. \\
Nun wird bei konstantem $U_\text{B}$ = 250 und 500V die Verschiebung D in Abhängigkeit vom Strom $I$ gemessen. 
\\
Dann wird mit niedriger Beschleunigungsspannung ($U_\text{B}$ = 150 - 200 V) experimentiert. Die Achse der Röhre wird in Nord-Süd-Richtung gedreht. Daraufhin wird sie in Ost-West-Richtung gedreht. Auf die bewegten Elektronen wirkt nun eine Kraft und sie werden in Y-Richtung abgelenkt. Um diese Kraft auszugleichen wird das Helmholtz-Feld mit einem Spulenstrom $I_\text{hor}$ eingeschaltet, sodass keine Ablenkung mehr erkennbar ist. Um die Totalintensität $B_\text{total}$ nun zu bestimmen wird zudem des Inklinationswinkel $\varphi$, der Winkel zwischen Horizontalebene und Richtung des Erdfeldes, aufgenommen. Hierzu wird das Inklinatorium um seine vertikale Achse gedreht bis die Magnetnadel parallel zur horizontalen Drechachse des Gerätes liegt. Daraufhin wird der Teilkreis um $90^\circ$ geschwnkt, woraufhin die Drehachse der Magnetnadel horizontal ist. Die NAdel zeigt nun in Feldrichtung und $\varphi$ kann abgelesen werden.