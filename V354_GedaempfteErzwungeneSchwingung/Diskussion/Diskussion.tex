\section{Diskussion}

Bei den Messungen gab es mehrere Fehlerquellen, welche die Ergebnisse verfälschen.
Sowohl das Ablesen des Oszilloskop, als auch die Innenwiderstände der Geräte, welche nicht berücksichtigt wurden, führen zu Messungenauigkeiten.
Durch längeren Betrieb der Apparaturen erwärmt sich der Schaltkreis, was dazu führt, dass die Widerstände kleiner werden. Somit verändern sich auch die messbaren Werte über die Zeit.
Scheinbar sind die Messungenauigkeiten erheblich, da große Fehler berechnet werden.

Die durch Messungen ermittelten Werte unterscheiden sich teils nur minimal, teils jedoch auch stark von den theoretischen Werten. Dies kann mit dem jeweiligen Versuch und den dabei verwendeten Methoden zusammen hängen.

