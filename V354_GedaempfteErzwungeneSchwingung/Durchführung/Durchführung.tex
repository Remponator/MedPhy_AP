\section{Durchführung}
\subsection{Zeitabhängigkeit der Amplitude}
Mit dem in Abbildung \ref{fig:a} gezeigten Aufbau wird die Zeitabhängigkeit der Amplitude des Schwingkreises gemessen. Da es in der Apparatur zwei wählbare Widerstände gibt, wird hier der kleinere Widerstand in der Schaltung verwendet. 
\\
Dazu wird mithilfe des Nadelimpulsgenerators ein einzelner Impuls in den gedämpften Schwingkreis gegeben. Auf dem Oszilloskop lässt sich dann eine Abnahme der Amplituden ablesen. Die Abstände der Amplituden und die dabei entstehende Zeitdifferenz werden notiert. 

\subsection{Dämpfungswiderstand}
Der zweite Versuchsteil wird mit dem Aufbau aus Abbildung \ref{fig:b} durchgeführt. Der regelbare Widerstand wird zu Beginn auf seinen maximalen Wert gestellt und dann langsam verkleinert. Dies geschieht so lange, bis auf dem Oszilloskop ein "Überschwingen" der Kurve sichtbar ist ($\frac{\text{d}U_\text{C}}{\text{dt}}$ > 0). Geschieht dies, so wurde $R_\text{ap}$ überschritten. 
Der Dämpfungswiderstand liegt somit genau da, wo das "Überschwingen" gerade wieder verschwindet.

\subsection{Frequenzabhängigkeit der Kondensatorspannung}
Nun wird der Aufbau aus Abbildung \ref{fig:c} verwendet. In diesem Versuchsteil wird der größere der beiden Widerstände verwendet.
\\
Mithilfe eines Sinusgenerators wird über $C$ die Spannung $U_\text{C}$ mithilfe des Tastkopfes gemessen. Dies geschieht für verschiedene Frequenzen der Wechselspannung. 
Da der Tastkopf einen eigenen Frequenzgang besitzt, muss zudem der Quotient aus Kondensaturspannung und Erregerspannung über den Tastkopf ermittelt werden ($\frac{U_\text{C}}{U}$).

\subsection{Frequenzabhängigkeit der Phasenverschiebung}
Für den letzten Versuchsteil wird der Aufbau aus \ref{fig:d} verwendet. 
\\
Auf dem Oszilloskop werden zwei Graphen mit einer Phasenverschiebung angezeigt.
Nun werden a und b (siehe \ref{fig:zuD}) gemessen.
\begin{figure}[H]
  \centering
  \includegraphics[height=5cm]{Durchführung/Grafiken/zuD.pdf}
  \caption{Messung der Phasenverschiebung, \cite{1}.}
  \label{fig:zuD}
\end{figure}
Über 
\begin{equation}
\varphi = \frac{a}{b} \cdot 2\pi
\end{equation}
ist dann die Phasenverschiebung gegeben.