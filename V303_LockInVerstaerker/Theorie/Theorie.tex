\section{Zielsetzung}

In diesem Versuch soll die Funktionsweise eines Lock-In-Verstärkers untersucht werden um das Prinzip der Rauschunterdückung kennenzulernen.

\section{Theorie}

Mithilfe eines Lock-In-Verstärkers können Signale mit starkem Rauschen gemessen werden. 
Der schematische Aufbau ist in Abbildung \ref{fig:Aufbau} dargestellt.
\begin{figure}[h]
  \centering
  \includegraphics[height=4cm]{Grafiken/Schaltung.pdf}
  \caption{Schematischer Aufbau eines Lock-In-Verstärkers. \cite{1}}
  \label{fig:Aufbau}
\end{figure}
Das zu untersuchende Nutzsignal $U_\text{sig}$ wird mit der Referenzfrequenz $\omega_0$ moduliert.  Zuerst entfernt der Bandpassfilter alle Frequenzanteile die viel höher oder niedriger sind als $\omega_0$. Der Mischer multipliziert nun $U_\text{sig}$ mit einem Referenzsignal $U_\text{ref}$ mit der Frequenz $\omega_0$. Damit die Phasen synchron sind ($\Delta \phi = 0$), wird die Phasenlage $\phi$ des Referenzsignals mithilfe des Phasenschiebers variiert.
Der Tiefpass ($\tau = RC >> \frac{1}{\omega_0}$) integriert nun $U_\text{sig} \times U_\text{ref}$  über mehrere Perioden, wodurch das Rauschen größenteils herausgemittelt wird. Am Ausgang wird somit eine Gleichspannung $U_\text{out} \propto U_0 cos\phi$ erreicht. 
Wird die Zeitkonstante $\tau = RC$ des Tiefpasses sehr groß gewählt, so wird die Bandbreite des Restrauschen $\nu = \frac{1}{\pi RC}$ sehr klein. Somit können Güten von bis zu Q = 100000 erreicht werden.
\\
\begin{figure}[h]
  \centering
  \includegraphics[height=7cm]{Grafiken/Spannung.pdf}
  \caption{Beispielverläufe der Spannungen. \cite{1}}
  \label{fig:Spannung}
\end{figure}
In Abbildung \ref{fig:Spannung} ist ein beispielhafter Signalverlauf einer Sinusspannung dargestellt mit
\begin{equation}
	U_\text{sig} = U_0sin(wt)
\end{equation}
Diese kann mit einer rechteckförmigen Referenzspannung mit gleicher Frequenz moduliert werden. Die Amplitude ist dabei auf 1 normiert.
Mithilfe einer Fourier-Transformation ergibt sich:
\begin{equation}
	U_\text{ref} = \frac{4}{\pi} \biggl(sin(\omega t) + \frac{1}{3}sin(3\omega t) + \frac{1}{5}sin(5\omega t) + ... \biggr)
\end{equation}
 Diese Reihe ist aus den ungeraden Harmonischen der Grundfrequenz $\omega$ zusammengesetzt.
 Die geraden Oberwellen von $\omega$ ergeben sich nun aus dem Produkt von Signal- und Referenzfrequenz:
 \begin{equation}
 	U_\text{sig} \times U_\text{ref} = \frac{2}{\pi}U_0\biggl(1-\frac{2}{3}cos(2\omega t)-\frac{2}{15}cos(4\omega t)-\frac{2}{35}cos(6\omega t)+...\biggr)
\end{equation}
Der Tiefpass unterdrückt nun diese Oberwelle, wodurch sich die zur Eingangsspannung proportionale Gleichspannung
\begin{equation}
	U_\text{out} = \frac{2}{\pi}U_0
\end{equation}
ergibt. Bei einer Phasendifferenz ergibt sich:
 \begin{equation}
  	U_\text{out}=\frac{2}{\pi}U_0cos(\phi)
\end{equation}

Den Maximalwert der Ausgangsspannung wird bei der Phase $\phi = 0$ erreicht.
