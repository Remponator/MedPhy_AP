\newpage
\section{Diskussion}
Der Versuch verifiziert die Funktionsweise des Lock-In-Verstärkers. Das erzeugte Rauschen wurde sehr gut hinausgefiltert, was in den Thermoabdrücken gut erkennbar ist. Die Abdrücke mit und ohne Rauschen erscheinen identisch. Dies spricht für eine hohe Güte des Lock.In-Verstärkers.
Auch ist die Phasenverschiebung ist sehr genau, was die Funktionsweise des Gleichrichters ebenfalls verifiziert. 
Zum Schluss sollte die Funktionsweise der Rauschunterdrückung des Lock-In-Verstärkers anhand einer Photodetektorschaltung überprüft werden.
Bei kurzem Abstand wurden die Messwerte stark durch Reflexion am Metall der Halterung verfälscht. Aus diesem Grund wurden die ersten Werte auf kurzem Abstand bei der grafischen Auswertung nicht beachtet. Bei den restlichen Werten ist ein quadratischer Abfall der Lichtintensität mit zunehmendem Abstand $r$ gut erkennbar. Dies lässt darauf schließen, dass der restliche Lichteinfall, ein Störsignal/Rauschen, erfolgreich herausgefiltert wurde. 
Ein maximaler Abstand konnte nicht ermittelt werden, da die Länge der Messvorrichtung dafür nicht ausreichte.
Abschließend kann gesagt werden, dass der hier untersuchte Lock-In-Verstärker sehr gut zur Messung stark verrauschter Signale verwendet werden kann.