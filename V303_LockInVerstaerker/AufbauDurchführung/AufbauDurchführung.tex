\section{Aufbau und Durchführung}
\begin{figure}[H]
  \centering
  \includegraphics[height=8cm]{Grafiken/Aufbau.pdf}
  \caption{Lock-In-Verstärker, \cite{1}.}
  \label{fig:Verstärker}
\end{figure}
In Abbildung \ref{fig:Verstärker} ist die Messaparatur zu sehen. 
Zu benutzen sind die Vorverstärker, die verschiedenen Filter (Hoch-, Tief- und Bandpass), der Lock-In-Detektor, ein Phasenverschieber, ein Funktionsgenerator, ein Rauschgenerator, sowie ein Tiefpass-Verstärker.
Zudem können mit einem Speicher-Oszilloskop alle Signale einzeln vermessen und abgespeichert werden.
\\
\\
Zu Beginn wird untersucht, welcher Ausgang des Funktionsgenerators variierbare Spannungsamplituden generiert und welcher Ausgang eine konstante Spannung und wie groß diese ist.
\\
\\
Danach wird die Schaltung aus Abbildung \ref{fig:Aufbau1} aufgebaut.
\begin{figure}[h]
  \centering
  \includegraphics[height=5cm]{Grafiken/Durchführung1.pdf}
  \caption{Schaltung zur Messung des Ausgangssignals in Phasenabhängigkeit, \cite{1}.}
  \label{fig:Aufbau1}
\end{figure}
Der Noise Generator ist dabei überbrückt, bzw. auf OFF. Es wird eine Sinusspannung $U_\text{sig}$ von ca. 1kHz und 10mV angelegt und mit einem Sinussignal ($U_\text{ref}$) gemischt. Dabei soll vor allem die Ausgangsspannung in Abhängigkeit zur Phasenverschiebung gemessen werden.
Die Messwerte werden hier jedoch, anders als in der Abbildung \ref{fig:Aufbau1} zu sehen, hinter dem Tiefpass abgegriffen, also nachdem sie durch den Tiefpass gelaufen sind.
Bei jeder Messung wurde ein Bild des Oszilloskopbildschirms bei verschiedenen Phasenverschiebungen aufgenommen (siehe Auswertung).
\newpage
Nun wird ein zusätzliches Rauschsignal mithilfe des Noise Generators angelegt. Dieses Rauschsignal soll in der Größenordnung der Signalspannung liegen. 
\\
\\
Zum Schluss wird eine Photodetektorschaltung wie in Abbildung \ref{fig:Aufbau2} aufgebaut. 
\begin{figure}[H]
  \centering
  \includegraphics[height=4cm]{Grafiken/Durchführung2.pdf}
  \caption{Schaltung zur Messung der Leuchtstärke der LED in Abhängigkeit zum Abstand $r$, \cite{1}.}
  \label{fig:Aufbau2}
\end{figure}
An die Leuchtdiode (LED) wird eine Rechteckspannung angelegt und somit mit 50 Hz bis 500 Hz zum blinken gebracht. 
Mithilfe einer Photodiode wird nun die Lichtintensität der LED gemessen. 
Die Messwerte am Lock-In-Verstärker werden jedoch, wie im vorherigen Versuchsteil, hinter dem Tiefpass abgegriffen.
Dabei wird der Abstand $r$ zur LED variiert um die Abhängigkeit der Intensität der LED zum Abstand $r$ und den maximalen Abstand $r_\text{max}$, bei dem das Licht nachgewiesen werden kann, zu ermitteln.