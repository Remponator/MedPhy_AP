\section{Auswertung}

\subsection{Bestimmung der Wellenlänge}

Die Wellenlänge des Lasers wird über die Anzahl der Maxima $z$ bei einer Verschiebung eines Spiegels um $\Delta d$ bestimmt.
Dazu werden die in Tabelle 1 aufgelisteten Wertepaare gemessen, wobei der angezeigte Wert eine Hebeluntersetzung von $1:5,017$ zur wirklichen Verschiebung des Spiegels hat.

\begin{table}[H]
    \begin{center}
      \label{tab:werte}
      \begin{tabular}{c|c|c}
        \hline
        \textbf{angezeigte} & \textbf{übersetzte} & \textbf{Anzahl der}\\
        \textbf{Verschiebung $[mm]$} & \textbf{Verschiebung  $\Delta d \;[mm]$} & \textbf{Maxima $z$}\\
        \hline
        5,0 & 0,997 & 3042 \\
        5,0 & 0,997 & 3069 \\
        5,0 & 0,997 & 3061 \\
        5,0 & 0,997 & 3093 \\
        5,0 & 0,997 & 3080 \\
        5,0 & 0,997 & 3113
      \end{tabular}
      \caption{Anzahl der Maxima bei Verschiebungen des Spiegels um $\Delta d$.}
    \end{center}
  \end{table}
  
Über die Gleichungen
\begin{equation}
    \label{eq:mean}
    \bar{x} = \frac{1}{N} \sum_{i=1}^{N} x_i \qquad \text{und} \qquad \Del{\bar{x}} = \sqrt{ \frac{\sum_{i=1}^{N} \left( x_i - \bar{x} \right)^2}{N\,(N-1)}} \: .
\end{equation}
wird der Mittelwert und der daraus resultierende Fehler der Anzahl der Maxima $z$ bestimmt.
Es ergibt sich ein Mittelwert von
\begin{equation}
    \bar{z} = 3076,34 \pm 10,17,   \notag
\end{equation}
woraus sich nach Gleichung \ref{eq:5} die Wellenlänge durch
\begin{equation}
    \lambda = \frac{2 \cdot \Delta d}{z}    \notag
\end{equation}
bestimmen lässt.
Über die Gau\ss{}sche Fehlerfortpflanzung nach
\begin{equation}
    \Delta \lambda = \sqrt{\left (- \frac{2 \cdot \Delta d}{z^{2}} \Delta z \right)^{2}} \notag
\end{equation}
wird die Wellenlänge auf
\begin{equation}
    \lambda = \SI{647,922 \pm 2,142}{nm}   \notag
\end{equation}
bestimmt.

\subsection{Bestimmung des Brechungsindizes}

Zur Bestimmung des Brechungsindizes werden die Werte aus Tabelle 2 aufgezeichnet.

\begin{table}[H]
    \begin{center}
      \label{tab:werte}
      \begin{tabular}{c|c}
        \hline
        \textbf{Druckunterschied $\Delta p \; [bar]$} & \textbf{Anzahl der Maxima $z$}\\
        \hline
        0,8 & 33 \\
        0,8 & 32 \\
        0,8 & 32 \\
        0,8 & 32 \\
        0,8 & 29 \\
        0,8 & 31
      \end{tabular}
      \caption{Anzahl der Maxima bei Veränderung des Brechungsindizes.}
    \end{center}
\end{table}

Dabei ergibt sich nach den Gleichungen \ref{eq:mean} ein Mittelwert und der dazugehörige Fehler von
\begin{equation}
    \bar{z} = 31,50 \pm 0,57.   \notag
\end{equation}
Mittels Gleichung \ref{eq:5}, der Herstellerangabe der Wellenlänge von $\lambda = \SI{635}{nm}$ und einer Länge des zu durchlaufenden Mediums von $b = \SI{50}{mm}$ ergibt sich
\begin{equation}
    \Delta n = 0,000200 \pm 0,0000004.    \notag
\end{equation}
mit der Gau\ss{}schen Fehlerfortpflanzung
\begin{equation}
    \Delta (\Delta n) = \sqrt{\left( \frac{\lambda}{2 b} \Delta z \right)^{2}}. \notag
\end{equation}

Zur Bestimmung des Brechungsindizes von Luft bei Normalbedingungen werden folgende Werte benötigt:
\begin{equation}
    p_0 = \SI{1,0132}{bar}  \notag
\end{equation}
\begin{equation}
    T_0 = \SI{273,15}{K}  \notag
\end{equation}
\begin{equation}
    T = \SI{293,15}{K}.  \notag
\end{equation}
Nach Gleichung \ref{eq:7} lässt sich nun der Brechungsindex für Luft bei Normalbedingungen auf
\begin{equation}
    n_\text{Luft} = 1,000272 \pm 0.000005   \notag
\end{equation}
bestimmen.
Der Fehler dabei ergibt sich auf der Gau\ss{}schen Fehlerfortpflanzung
\begin{equation}
    \Delta n = \sqrt{\left (\frac{T}{T_0} \frac{p}{\Delta p} \Delta (\Delta n) \right )^{2}}.   \notag
\end{equation}
\section{Diskussion}
Es fällt auf, dass der berechnete Wert der Wellenlänge eine gute Übereinstimmung mit der Herstellerangabe hat.
Der experimentelle Wert $\lambda_\text{exp} = \SI{647,922 \pm 2,142}{nm}$ hat dabei eine Abweichung von $2 \%$ zur Herstellerangabe von $\lambda_\text{Her} = \SI{635}{nm}$.\\
Des Weiteren fällt auf, dass der Unterschied des Brechungsindizes sehr gering ist.
Das kann trotz des hohen Druckunterschiedes auf die systematischen Fehler der Messgeräte zurückgeführt werden.
Trotzdem kann auf gleichmäßige Messungen geschlossen werden, da die Fehler der Mittelwerte jeweils relativ gering sind.\\
Der bestimmte Brechungsindex von Luft bei Normalbedingungen liegt bei $n_\text{exp} = 1,000272 \pm 0.000005$ und hat zum Literaturwert von $n_\text{Lit} = 1,000292$ eine Abweichung von $0,002 \%$ und ist somit vernachlässigbar gering.
