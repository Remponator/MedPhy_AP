\section{Aufbau und Durchführung}
\label{sec:Aufbau}

\begin{figure}[H]
    \centering
    \includegraphics[height=8cm]{Aufbau/Aufbau.pdf}
    \caption{Versuchsaufbau.}
    \label{fig:Aufbau}
\end{figure}

In Abbildung \ref{fig:Aufbau} ist der Versuchsaufbau abgebildet. Er besteht aus dem Michelson-Interferometer, einem Laser und der Messzelle mit Impulszähler. Für den zweiten Versuchsteil ist zudem eine Vakuumpumpe mit Manometer angeschlossen mit der der Druck in der Laufbahn des Lasers verändert werden kann. \\
Zu Beginn wird das Michelson-Interferometer so justiert, dass an der Messzelle kreisförmige Interferenzen zu sehen sind. Dazu wird vor die gebündelten Laserstrahlen eine Zerstreuungslinse gestellt. \\
Nun wird der verschiebbare Spiegel um 5 mm kontinuierlich verschoben und dabei die Licht-Impulse an der Messzelle aufgenommen. Dies wird sechs Mal wiederholt. \newline
Zur Bestimmung des Brechungsindex der Luft wird ein Bereich in der Laufbahn des Lasers evakuiert und dann wieder langsam mit Luft befüllt. Dabei werden erneut die Impulse gezählt. Auch hier wird die Messung sechs mal wiederholt.