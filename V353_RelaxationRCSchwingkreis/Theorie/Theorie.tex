\section{Theorie}
\subsection{Allgemeine Relaxationsgleichung}

Von Relaxation wird gesprochen, wenn ein System nach Auslenkung ,nicht-oszilllatorisch in seinen Ausgangszustand zurückkehrt. 
Hier wird ein RC-Kreis, also eine einfache Schaltung aus Widerstand und Kondensator betrachtet.
Die Änderungsgeschwindigkeit zur Zeit t der Größe A ist proportional zu dessen Abweichung vom Endzustand A($\infty$)
\begin{equation}
	\frac{dA}{dt} = c [A(t) - A(\infty)]
\end{equation}
Durch Integration folgt:
\begin{equation}
	A(t) = A(\infty) + [a(0) - A(\infty)] e^{ct}.
\end{equation}
Hier muss c < 0 sein, da A sonst nicht beschränkt ist.

\subsection{Entladung}
\begin{figure}[h]
  \centering
  \includegraphics[height=5cm]{Grafiken/Entladung.pdf}
  \caption{Entladung (1) und Aufladung (2) eines Kondensators über einen Widerstand, \cite{1}.}
  \label{fig:Entladung}
\end{figure}
Zwischen den beiden Platten eines Kondensators mit Kapazität $C$ und Ladung $Q$ liegt eine Spannung $U_\text{C}$ an:
\begin{equation}
	U_\text{C}
\end{equation}
Nach dem ohmschen Gesetz folgt der Strom $I$ durch den Widerstand $R$:
\begin{equation}
	I = \frac{U_\text{C}}{R}
\end{equation}
Zeitlich verändert sich die Ladung $Q$ auf den Platten mit:
\begin{equation}
	dQ = - I dt
\end{equation}
Somit folgt aus (3),(4) und (5):
\begin{equation}
	Q(t) = Q(0)\cdot e^{(-\frac{t}{RC})}
\end{equation}

\subsection{Aufladung}
Der Aufladevorgang lässt sich mit der Gleichung
\begin{equation}
	Q(t) = CU_0(1-e^{(-\frac{t}{RC})})
\end{equation}
beschreiben. RC ist dabei die Zeitkonstante des Relaxationsvorganges. Sie beschreibt die Geschwindigkeit, in der das System dem Endzustand Q($\infty$) zugeht. Während der Zeit $\increment$T = RC ändert sich die Ladung auf dem Kondensator um
\begin{equation}
	\frac{Q(t = RC)}{Q(0)} = \frac{1}{e} \approx 0,368 \notag
\end{equation}
10$\%$ des Ausgangswertes sind nach $\increment$T = 2,3 RC noch vorhanden, 1 $\%$ nach 4,6 RC.

\subsection{Relaxation bei periodischer Auslenkung}
Es wird eine äußere Wechselspannung U(t) angelegt:
\begin{equation*}
	U(t) = U_0 cos\omega t
\end{equation*}
\begin{figure}[h]
  \centering
  \includegraphics[height=5cm]{Grafiken/Wechsel.pdf}
  \caption{Schaltung eines RC-Kreises mit angelegter Wechselspannung, \cite{1}.}
  \label{fig:Wechsel}
\end{figure}
Ist $\omega$ klein genug, also $\omega$<<$\frac{1}{RC}$, so wird die Spannung $U_\text{C}$ gleich U(t) sein. Je höher die Frequenz, desto mehr verschiebt sich die Auf-/Entladung des Kondensators über den Widerstand hinter den Verlauf der Generatorspannung. Es entsteht also eine Phasenverschiebung $\varphi$ zwischen den beiden Spannungen. Zudem nimmt die Amplitude A der Kondensatorspannung ab. Veranschaulicht wird der Vorgang in Abbildung \ref{fig:Wechsel}.
Die ausgehende Wechselspannung wird nun als 
\begin{equation}
	U_\text{C}(t) = A(\omega)cos(\omega t + \varphi(\omega))
\end{equation}
beschrieben.
Zudem gilt:
\begin{equation}
	I(t) = \frac{dQ}{dt} = C \frac{dU_\text{C}}{dt}
\end{equation}
Über die bisherigen Gleichungen und dem Kirchhoff'schen Gesetz folgt dann:
\begin{equation}
	A(\omega) = \frac{U_0}{\sqrt{1+\omega^2 R^2 C^2}}
\end{equation}
Diese Gleichung stellt eine Beziehung zwischen Kondensatorspannungs-Amplitude und der Kreisfrequenz der Erregerspannung auf. Für $\omega$ $\rightarrow \infty$ läuft A($\omega$) gegen 0 und für $\omega$ $\rightarrow$ 0 gegen $U_0$. Zudem ist A(1/RC) = $U_0$ / $\sqrt{2}$. 
Wegen diesen Zusammenhängen nutzt man RC-Kreise in der elektrischen Schaltungstechnik als Tiefpässe, da sie Frequenzen, die klein gegen 1/RC laufen ungehindert hindurchlassen und hohe Frequenzen $\omega$ >> 1/RC schlecht durchlassen. 

\subsection{RC-Kreis als Integrator}
Ein RC-Kreis kann unter bestimmten Bedingungen auch als Integrator verwendet werden. $\omega$ muss dabei >> 1/RC sein.
Dann folgt aus
\begin{equation*}
	U(t) = RC \frac{dU_\text{C}}{dt}
\end{equation*}
unter besagter Voraussetzung:
\begin{equation}
	U_\text{C}(t) = \frac{1}{RC} \int_0^t U(t')dt'
\end{equation}