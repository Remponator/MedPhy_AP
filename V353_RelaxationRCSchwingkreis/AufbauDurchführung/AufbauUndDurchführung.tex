\section{Aufbau und Durchführung}
\subsection{Bestimmung der Zeitkonstante}
Zu Beginn wird die Zeitkonstante gemessen. Dazu wird die Schaltung wie in Abbildung  \ref{fig:a} aufgebaut.
\begin{figure}[H]
  \centering
  \includegraphics[height=5cm]{Grafiken/a.pdf}
  \caption{Aufbau der Schaltung zur Bestimmung der Zeitkonstante, \cite{1}.}
  \label{fig:a}
\end{figure}
Die Schaltung besteht aus einer Kapazität C, einem Widerstand R, einem Oszilloskop und einem Spannungsgenerator, welcher eine Rechteckspannung $U_0$(t) erzeugt. Auf dem Oszilloskop sind dann die Auflade- und Entladevorgänge des Kondensators zu sehen. 

\subsection{Bestimmung der Spannungsamplitude}
Nun wird die Spannungsamplitude der Kondensatorspannung gemessen. Erneut wird die Schaltung aus Abbildung \ref{fig:a} verwendet.
Der Spannungsgenerator liefert diesmal eine Sinusspannung $U_0$(t). die Frequenz f der Spannung wird logarithmisch erhöht und die dazugehörigen Amplituden A notiert.

\subsection{Bestimmung der Phasenverschiebung}
In der drittenMessreihe wird die Phasenverschiebung der Spannung am Kondensator und der erregenden Spannung untersucht. Dazu wird die Schaltung aus Abbildung \ref{fig:c} verwendet.
\begin{figure}[H]
  \centering
  \includegraphics[height=5cm]{Grafiken/c.pdf}
  \caption{Aufbau der Schaltung zur Bestimmung der Phasenverschiebung, \cite{1}.}
  \label{fig:c}
\end{figure}
Der Spannungsgenerator erzeugt eine Sinusspannung, wodurch auf dem Oszilloskop zwei Spannungsverläufe zu sehen sind, der Spannungsverlauf am Kondensator $U_\text{C}$(t) und der generierten Spannung $U_\text{G}$(t). Die Frequenzen werden logarithmisch erhöht und die Differenzen der Nullstellen beider Kurven und die Wellenlänge der Kurve $U_0(t)$ notiert.

\subsection{Bestätigung der Integratorfunktion}
Zum Schluss wird die Schaltung aus Abbildung \ref{fig:d} aufgebaut. 
\begin{figure}[H]
  \centering
  \includegraphics[height=5cm]{Grafiken/d.pdf}
  \caption{Aufbau zur Bestätigung der Integratorfunktion, \cite{1}.}
  \label{fig:d}
\end{figure}
Auf dem Oszilloskop sind jetzt der Spannungsverlauf $U_0(t)$ der generierten Spannung und der integrierte Spannungsverlauf $U_\text{C}(t)$ sichtbar. 
Der Reihe nach wird eine Rechteckspannung, eine Dreiecksspannung und eine Sinusspannung angelegt und die Thermoabdrücke des Oszilloskops abgespeichert. 