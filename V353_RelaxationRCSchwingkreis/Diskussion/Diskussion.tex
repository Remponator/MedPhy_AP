\section{Diskussion}

Die bestimmten Werte der Zeitkonstane RC unterscheiden sich stark. Dies lässt auf große Ungenauigkeiten in den einzelnen Messmethoden schließen.
Die Graphen aus den experimentellen Werten ähneln stark den theoretischen Graphen.
Wie erwartet nimmt die Amplitude der Kondensatorspannung exponentiell ab und die Phasenverschiebung bleibt im Bereich von 0 und $\frac{\pi}{2}$.
Vermutlich führen Ableseschwierigkeiten am Oszilloskop zu Ungenauigkeiten und Abweichungen, die vor allem in Abbildung 8 sichtbar werden.
Das RC-Glied kann gut als Integrator verwendet werden, da die Abbilder des Oszilloskopbildschirms die jeweiligen Funktionen zeigen, wie sie vorher bestimmt worden sind.

\section{Literatur}

[1] TU Dortmund. Versuchsanleitung zum Experiment V353 - Das Relaxationsverhalten eines RC-Kreises. 2018.