\section{Aufbau}
Der Versuchsaufbau ist in Abbildung \ref{fig:Aufbau} dargestellt.
\begin{figure}[h]
  \centering
  \includegraphics[height=5cm]{Grafiken/Aufbau.pdf}
  \caption{Versuchsaufbau zur Messung der Wärmeleitung, \cite{1}.}
  \label{fig:Aufbau}
\end{figure}
Es handelt sich um ein Peltier Element, auf dem vier Probestäbe aus drei verschiedenen Stoffen montiert sind. An jedem Probestab befinden sich zwei Thermoelemente, mit denen die Temperaturen $T_\text{nah}$ und $T_\text{fern}$ gemessen werden. Diese Messwerte werden über ein Temperatur Array an einen Datenlogger Xplorer GLX übergeben. Über eine Spannungsquelle werden die Probestäbe erhitzt. 