\section{Zielsetzung}
In diesem Versuch soll die Wärmeleitung von Aluminium, Messing und Edelstahl und die stoffspezifischen Wärmeleitfähigkeiten untersucht werden.
Dazu wird eine statische und eine dynamische Methode durchgeführt (siehe Durchführung).

\section{Theorie}
Jeder Körper strebt einem Temperaturgleichgewicht entgegen. Dieses Temperaturgleichgewicht wird durch Konvektion, Wärmestrahlung oder Wärmeleitung entlang des Temperaturgefälles erreicht. 
Dieser Vorgang kann bei einem Stab der Länge L, einer Querschnittsfläche A, der Materialdichte $\rho$ und der spezifischen Wärme $c$ beschrieben werden als
\begin{equation}
	\text{d}Q = -\kappa \frac{\partial T}{\partial x} \text{d}t .
\end{equation}
Die Wärmemenge Q fließt also immer in Richtung abnehmender Temperatur.
$\kappa$ beschreibt die (materialabhängige) Wärmeleitfähigkeit. 
Für die Wärmestromdichte $j_\text{\omega}$ gilt
\begin{equation}
	j_\text{\omega} = -\kappa \frac{\partial T}{\partial x}
\end{equation}
Daraus ergibt sich mithilfe der Kontinuitätsgleichung die eindimensionale Wärmeleitungsgleichung
\begin{equation}
	\frac{\partial T}{\partial t} = \frac{\kappa}{\rho c} \frac{\partial^2 T}{\partial x^2}
\end{equation}
Die Wärmeleitungsgleichung beschreibt die räumliche und zeitliche Entwicklung der Temperaturverteilung im Körper. 
Der Term $\sigma_\text{T} = \frac{\kappa}{\rho c}$ beschreibt die Temperaturleitfähigkeit. 
Bei periodischer Erwärmung / Abkühlung des Stabes, entsteht eine Temperaturwelle mit der Form
\begin{equation}
	T(x,t) = T_\text{max} e^{-\sqrt{\frac{\omega \rho c}{2\kappa}}x}cos\biggl(\omega t - \sqrt{\frac{\omega \rho c}{2\kappa}}x \biggr)
\end{equation}
Die Phasengeschwindigkeit dieser Welle ist
\begin{equation}
	v = \frac{\omega}{k} = \frac{\omega}{\sqrt{\frac{\omega \rho c}{2\kappa}}} = \sqrt{\frac{2\kappa \omega}{\rho c}} .
\end{equation}
Durch Umformung und dem Verhältnis der Amplituden $A_\text{nah}$ und $A_\text{fern}$ der Welle an den Messstellen $x_\text{nah}$ und $x_\text{fern}$ ergibt sich
\begin{equation}
	\kappa = \frac{\rho c (\increment x)^2}{2\increment t \cdot ln(A_\text{nah} / A_\text{fern})}
\end{equation}
Dies ist die Wärmeleitfähigkeit. 
$\increment x$ ist dabei der Abstand der beiden Messstellen und $\increment t$ die Phasendifferenz der Temperaturwelle zwischen den Messstellen.