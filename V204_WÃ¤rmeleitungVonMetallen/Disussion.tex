\section{Diskussion}
Allgemein ließ der Versuch darauf schließen, dass verschiedene Stoffe unterschiedliche Wärmeleitfähigkeiten besitzen.
Das kann daran erkannt werden, dass sich die verschiedenen Stoffe unterschiedlich schnell erwärmen und jeweils eine andere Höchsttemperatur anstreben.
So hat Aluminium beispielsweise die höchste Wärmeleitfähigkeit und erwärmt sich am schnellsten und Edelstahl am langsamsten.
Außerdem fällt auf, dass die übertragene Wärme auch von der Querschnittsfläche der Probe abhängt.
Eine größere Querschnittsfläche ist gleichbedeutend mit einer größeren übertragenen Wärmemenge.
Bei der statischen Methode fällt auf, dass bei Messing die übertragene Wärmemenge mit der Zeit abnimmt und beim Edelstahl ansteigt.
Dies lässt sich auf die verschiedenen Materialeigenschaften der Stoffe schließen.
In Tabelle \ref{tab:Dis} werden die Literaturwerte und die gemessenen Werte für die Wärmeleitfähigkeit verglichen.
Dabei fällt auf, dass bei Messing der geringste Fehler auftritt und bei Edelstahl der größte Fehler.
Es kann geschlossen werden, dass die dyamische Methode mit steigender Schwingungsdauer  aufgrund der äußeren Einflüsse ungenauer wird.
Aus Tabelle \ref{tab:Dis} lässt sich schließen, dass die geringe Wärmeisolierung keine optimalen Werte zulässt.
Außerdem sind die Werte durch die verschiedene Beschaffenheit des Materials und der ungleichen Anfangstemperatur verändert.
Außerdem können äußere Einflüsse wie ein Windstoß das Messergebnis manipulieren.

\begin{table}[H]
    \begin{center}
      \caption{Ergebnisse der Messungen.}
      \label{tab:Dis}
      \begin{tabular}{c|c|c|c|c|c} 
        \textbf{Material} & \textbf{$\kappa_\text{Lit}/ \frac{W}{m \cdot K}$} &  \textbf{$\kappa_\text{Exp}/ \frac{W}{m \cdot K}$} & \textbf{absoluter Fehler /$ \frac{W}{m \cdot K}$} & \textbf{relativer Fehler / $\%$}\\
        \hline
        \text{Messing}     & 105 & 101,808 \pm 4,535 & 4,192   & 3,99 \\
        \text{Aluminium}   & 220 & 176,969 \pm 7,631 & 43,031  & 19,56 \\
        \text{Edelstahl}   & 20 &  14,207 \pm 0,221 &  5,793   & 28,97 \\
      \end{tabular}
    \end{center}
\end{table}
