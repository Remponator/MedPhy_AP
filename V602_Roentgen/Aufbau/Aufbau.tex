\section{Aufbau}
\label{sec:Aufbau}
Der Versuchsaufbau ist in Abbildung \ref{fig:Aufbau} abgebildet. 
\begin{figure}[h]
    \centering
    \includegraphics[height=6cm]{Aufbau/Aufbau.pdf}
    \caption{Versuchsapparatur zur Bestimmung der Röntgenemission- und absorption.}
    \label{fig:Aufbau}
\end{figure}
Grundsätzlich besteht er aus einer Kupfer-Röntgenröhre, einem LiF-Kristall und einem Geiger-Müller-Zählrohr. Bei der restlichen Apparatur handelt ist sich um einen PC, mithilfe dessen die verschiedenen Winkel eingestellt werden können. 

\section{Durchführung}
\label{sec:Durchführung}
Zur Versuchsdurchführung wird das Programm $measure$ verwendet. Für jede Messung wird eine Beschleunigungsspannung von $U_\text{B} = 35 \text{kV}$ und ein Emissionstrom von $I = 1 \text{mA}$ eingestellt. \\
Mithilfe der ersten Messung wird die Bragg-Bedingung überprüft. Hierzu wird im Programm die Messart $Spektren$ verwendet und bei einem festen Kristallwinkel von $\Theta = 14^\circ$ der Winkelbereich von $\alpha = 26^\circ$ bis $30^\circ$ gewählt. Der Winkelzuwachs bei einer Integrationszeit von $\increment t = 10 \text{s}$ beträgt dabei $0,1^\circ$.\\
Als nächstes wird das Emissionsspektrum der Röntgenröhre analysiert. Hierzu wird der Kopplungsmodus 2:1 eingestellt. Der Winkelbereich beträgt nun $\alpha = 4^\circ$ bis $26^\circ$ bei einer Integrationszeit von 5 s und einem Winkelzuwachs von $0,2^\circ$.\\
Zuletzt werden die Absorptionsspektren, genauer die K, bzw. L-Kanten, verschiedener Materialien aufgetragen. Als Messbereich wird dabei eine Umgebung von $\pm 1$ um den jeweiligen Braggwinkel $\Theta_\text{K/L}^\text{Lit}$ zum Literaturwert $E_\text{K/L}^\text{Lit}$ der K-, bzw L-Kante. Der Winkelzuwachs beträgt 0,1 s bei einer Integrationszeit von 20 s. Diese Messung wird für vier leichte und eine schweres Absorbermaterialien durchgeführt.