\section{Diskussion}
Zunächst lässt sich sagen, dass die Bragg-Bedingung aufgrund der geringen Abweichung als erfüllt angesehen werden kann.
Es wurde anstatt eines Maximums bei einem Winkel von $\theta_\text{theo} = \SI{14}{°}$ ein Winkel von $\theta_\text{exp} = \SI{14.35}{°}$ gemessen.
Das ergibt eine Abweichung von $2,5 \%$.

Für die drei Abschirmkonstanten ergeben sich
\begin{equation}
    \sigma_1 = 3,375  \qquad \text{,} \qquad \sigma_2 = 12,727 \qquad \text{und} \qquad \sigma_3 = 28,545. \notag
\end{equation}

Des Weiteren wurden die Abschirmkonstanten der K-Linien von Gallium, Zink, Zirkonium und Brom bestimmt.
Die dazu notwendigen Energien sind in Tabelle \ref{tab:ene} aufgelistet.

\begin{table}[H]
    \begin{center}
      \caption{Messergebnisse der Energien der K-Kante.}
      \label{tab:ene}
      \begin{tabular}{c|c|c|c} 
        \textbf{Material} & \textbf{$E_\text{K,exp}$} & \textbf{$E_\text{K,theo}$} & \textbf{relative Abweichung [$\%$]}\\
        \hline
          \text{Gallium}    & 10,372 & 10,377 & 0,1\\
          \text{Zink}       & 9,670 & 9,668 & 0,2 \\
          \text{Zirkonium}  & 17,939 & 18,008 & 0,4\\
          \text{Brom}       & 13,507 & 13,483 & 0,2 \\
      \end{tabular}
    \end{center}
\end{table}

Für alle experimentell bestimmten Energien der K-Kanten lässt sich sagen, dass die Abweichungen sehr gering sind und die Messung somit aussagekräftig ist.
Für die resultierenden Abschirmkonstanten, die in Tabelle \ref{tab:werte} dargestellt sind, ergaben sich aber durch Näherungen bei der Rechnung größere Fehler.
Dies ist auf Ablesefehler und auf die jeweiligen Absorber zurückzuführen.
\begin{table}[H]
    \begin{center}
      \caption{Messergebnisse der Abschirmkonstanten.}
      \label{tab:werte}
      \begin{tabular}{c|c|c|c} 
        \textbf{Material} & \textbf{$\sigma_\text{1,exp}$} & \textbf{$\sigma_\text{1,theo}$} & \textbf{relative Abweichung [$\%$]}\\
        \hline
          \text{Gallium}    & 3,162 & 3,151 & 0,35 \\
          \text{Zink}       & 3,134 & 3,560 & 11,9 \\
          \text{Zirkonium}  & 3,251 & 4,080 & 21,2 \\
          \text{Brom}       & 3,170 & 3,830 & 17,2 \\
      \end{tabular}
    \end{center}
\end{table}

\subsection{Der schwere Absorber}
Für den schweren Absorber, in diesem Fall Wismut mit einer Kernladungszahl von $z = 83$ wurden die L-Kanten untersucht.
In dem vorliegenden Messbereich wurden die $L_\text{II}$ und die $L_\text{III}$-Kanten untersucht.
Dabei wurden die Werte
\begin{equation}
    E_\text{LII} = \SI{15,470}{keV} \qquad \text{und} \qquad E_\text{LIII} = \SI{13,212}{keV}
\end{equation}
gemessen und mit den Literaturwerten
\begin{equation}
    E_\text{LII,Lit} = \SI{15,711}{keV} \qquad \text{und} \qquad E_\text{LIII,Lit} = \SI{13,419}{keV}
\end{equation}
verglichen.
Dabei ergaben sich Abweichungen von jeweils $1,6 \%$.
Diese Abweichung lassen sich auf Fehler statistischen Ursprungs, des Ablesens und den Absorber zurückführen.
Trotzdem haben die beiden gemessenen Werte eine hohe Aussagekraft.

\subsection{Das Moseleysche Gesetz}
Bei der Untersuchung des Moseleyschen Gesetzes wurde die Rydbergenergie auf
\begin{equation}
    R_\infty = \SI{12,769}{eV}  \notag
\end{equation}
bestimmt.
Zu dem Literaturwert von $R_\text{$\infty$,Lit} = \SI{13,6}{eV}$ hat der ermittelte Wert eine Abweichung von $6,2 \%$.
Es lässt sich sagen, dass die lineare Regression sehr genau war, da der Fehler 5 Größenordnungen unter dem eigentlichen Wert lag.
Außerdem lässt sich sagen, dass der Fehler der Rydbergenergie einerseits auf die geringe Anzahl der Messwerte zurückzuführen ist.
Andererseits natürlich auch auf die Tatsache, dass das eigentliche Ergebnis der Steigung aus der linearen Regression noch quadriert werden musste, wodurch sich auch die Abweichung quadratisch vergrö\ss{}ert.
