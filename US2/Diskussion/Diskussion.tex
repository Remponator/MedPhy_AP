\section{Diskussion}
Bei der Untersuchung des Acrylblocks liefern beide Scan-Methoden ähnliche Werte. Es ist jedoch anzunehmen, dass die Methode des A-Scans für die Bestimmung der Lage der Löcher genauer ist, da bei dieser Messmethode keine Fehler durch zu schnelles Bewegen der Sonde auftreten können, wie bei dem B-Scan.\\
Zudem fällt auf, dass beim B-Scan für die beiden Seiten unterschiedliche Ergebnisse für die Löchergrößen und -lagen auftreten. Dies kann wie bereits erwähnt aus zu schnellen Bewegen der Sonde oder ungenauem Ablesen der Grafik resultieren. Auch konnte bei dem B-Scan ein Loch nicht erfasst werden, da es durch ein darüber liegendes Loch verdeckt wurde. 
\\
Bei der Untersuchung des Auflösungsvermögens bei verschiedenen Frequenz kann zusammengefasst werden, dass mit wachsender Frequenz auch die Auflösung schärfer wird. Die Peaks werden dabei jedoch kleiner, weshalb die Frequenz vermutlich nicht zu hoch gewählt werden kann. Bei einer Frequenz von 1MHz konnten die beiden untersuchten Löchern nicht mehr unterschieden werden.
\\
Mithilfe des TM-Scans konnte die Herzfrequenz, sowie das Herzzeitvolumen sehr gut berechnet werden. 
Für das Herzzeitvolumen ergibt sich ein Wert von
\begin{equation*}
	\text{HZV} = 3979 \frac{\text{mL}}{\text{min}}  .
\end{equation*}
Der Normalwert in Ruhe eines gesunden Erwachsenen liegt bei $4500-5000 \frac{\text{mL}}{\text{min}}$ \cite{3}.