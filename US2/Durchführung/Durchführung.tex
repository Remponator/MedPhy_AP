\section{Durchführung}

\subsection{Der A-Scan}
Zunächst sollen mit Hilfe des A-Scans die Störstellen des Acrylblocks, der in Abbildung \ref{fig:acryl} dargestellt ist, identifiziert werden.
Dazu wird mit einer $\SI{1}{MHz}$-Sonde die Lage der Bohrungen untersucht.
Dabei und bei den folgenden Messungen sollte Wasser als Kontaktmittel verwendet werden, da Luft einen sehr hohen Absorptionskoeffizienten hat.
Die Sonde wird über die Störstellen gesetzt und die Laufzeit gemessen.
Danach wird die Messung von der anderen Seite wiederholt um die Dicke der Störstellen zu bestimmen.
Es muss die Laufzeitkorrektur aufgrund des Kontaktmittels berücksichtig werden.

\subsection{Das Auflösungsvermögen}
In der nächsten Messung werden Sonden von $\SI{1}{MHz}$, $\SI{2}{MHz}$ und $\SI{4}{MHz}$ verwendet.
Dann werden die Störstellen 1 und 2 jeweils untersucht und Messungen mit dem A-Scan aufgenommen.

\subsection{Der B-Scan}
Bei dieser Messung wird mit einer $\SI{2}{MHz}$-Sonde der B-Scan durchgeführt.
Dabei wird die Sonde langsam und konstant über den Acrylblock bewegt um ein gleichmä\ss{}iges Bild zu erhalten.
Dann werden jeweils die Messzeitdifferenzen der Störstellen notiert.
Daraufhin wird der Acrylblock umgedreht und die Messung wiederholt.

\subsection{Der TM-Scan}
Es wird ein einfaches Herzmodell untersucht, das aus einem Doppelgefä\ss{} und einer beweglichen Membran besteht.
Dabei kann die Membran eigenständig durch einen Gummiball bewegt werden.
Aus der Messung werden dann aus dem enddiastolischen und das endsystolischen Volumen das Herzvolumen bestimmt.
Dazu wird der obere Behälter zur Hälfte mit Wasser gefüllt.
Mit einer $\SI{2}{MHz}$-Sonde wird dann bei regelmä\ss{}igem Aufpumpen der Membran ein TM-Scan des Herzens gemacht.
Dabei sollte die Sonde auf der Wasseroberfläche angebracht sein.

\begin{figure}[h]
    \centering
    \includegraphics[height=5cm]{Durchführung/Objekt.pdf}
    \caption{Der Acrylblock. \cite{US2}}
    \label{fig:acryl}
\end{figure}