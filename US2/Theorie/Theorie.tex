\section{Zielsetzung}
Im Folgenden werden die verschiedenen Scan-Verfahren und das damit verbundene Auflösungsvermögen untersucht.
Außerdem soll die Sonographie als Diagnosemethode an einem speziellen Beispiel untersucht werden.

\section{Theorie}
Der Ultraschall erstreckt sich in einem Frequenzbereich von $\SI{20}{kHz}$ bis $\SI{1}{GHz}$.
Unterhalb dieses Bereichs liegt der hörbare Bereich für Menschen und unter $\SI{20}{Hz}$ der Infraschallbereich.
Oberhalb des Bereichs liegt der Hyperschallbereich.

Au\ss{}erdem breitet sich Schall parallel zur Bewegungsrichtung aus und ist somit eine longitudinale Welle der Form
\begin{equation}
    p(x,t) = p_0 + v_0 Z cos(\omega t - kx), \notag
\end{equation}
wobei $Z$ die akustische Impedanz darstellt, die das Produkt aus der Dichte des Mediums und der Schallgeschwindigkeit im Medium, in dem sich die Welle fortbewegt, ist.
Des Weiteren sorgen Druckschwankungen für die Ausbreitung der Welle.
Diese Theorie gilt für Gase und Flü\ss{}igkeiten, wobei sich für Flü\ss{}igkeiten eine Schallgeschwindigkeit von
\begin{equation}
    \label{eq:1}
    c_\text{Fl} = \sqrt{\frac{1}{\kappa \rho}}
\end{equation}
einstellt, wobei $\kappa$ die Kompressibilität darstellt.

In Festkörpern breiten sich durch Schubspannungen neben den longitudinalen Wellen auch transversale Wellen aus.
Dadurch ergibt sich für die Schallgeschwindigkeit in einem Festkörper
\begin{equation}
    \label{eq:2}
    c_\text{Fe} = \sqrt{\frac{E}{\rho}},
\end{equation}
wobei $E$ das Elastizitätsmodul ist.

Ein weiter Aspekt der Schallausbreitung ist, dass die Schallwellen durch Absorption teilweise Energie abgeben, womit auch die Anfangsintensität $I_0$ nach
\begin{equation}
    I (x) = I_0 \cdot e^{- \alpha x}  \notag
\end{equation}
abnimmt.
Dabei ist $\alpha$ der Absorptionskoeffizient der Schallamplitude.

Darüber hinaus kommt es an Grenzflächen von zwei verschiedenen Medien zur Reflexion und Transmission der Schallwellen.
Dabei wird über die akustischen Impedanzen der beiden angrenzenden Medien der Relexionskoeffizient nach
\begin{equation}
    R = \left(\frac{Z_1 - Z_2}{Z_1 + Z_2} \right)^2 \notag
\end{equation}
berechnet, wobei der Transmissionskoeffizient dann $T = 1- R$ beträgt.

Bei der Erzeugung von Ultraschall werden Kristalle wie Quarze verwendet und zu Schwingungen angeregt.
Beim reziproken piezo-elektrischen Effekt wird ein piezoelektrischer Kristall in ein elektrisches Wechselfeld gebracht, sodass dieser zum Schwingen angeregt wird.
Dabei werden Ultraschallwellen ausgesendet, die bei der Resonanzfrequenz des Kristalls hohe Schwingungsamplituden aufweisen.
Darüber hinaus können die Kristalle auch als Schallempfänger verwendet werden, indem Schallwellen auf sie gerichtet werden und die Kristalle dadurch zum Schwingen angeregt werden.

Es werden zwei Methoden in der Ultraschalltechnik verwendet, die in Abbildung \ref{fig:1} dargestellt sind.
\begin{figure}[h]
    \centering
    \includegraphics[height=11cm]{Theorie/Verfahren.pdf}
    \caption{Prinzipieller Versuchsaufbau. \cite{US2}}
    \label{fig:1}
\end{figure}

Das Durchschallungsverfahren beschreibt die Technik, bei der auf der einen Seite der Probe der Sender und auf der anderen Seite der Empfänger angebracht werden.
So kann bestimmt werden, ob sich eine Fehlstelle in der Probe befindet, da am Empfänger eine abgeschwächte Intensität gemessen wird.
Dabei kann der Ort der Fehlstelle aber nicht genau bestimmt werden.
Das Impuls-Echo-Verfahren beschreibt die Messtechnik, bei der der Sender und Empfänger auf der gleichen Seite angebracht werden.
Dabei werden die ausgesendeten Ultraschallwellen an Fehlstellen reflektiert und es kann durch die Laufzeitbestimmung nach 
\begin{equation}
    \label{eq:3}
    s = \frac{1}{2} c t
\end{equation}
der genaue Ort der Fehlstelle bestimmen.

Dazu kommt, dass es drei verschiedene Darstellungsarten der Laufzeitdiagramme gibt.
Beim A-Scan oder auch Amplituden-Scan werden die Echoamplituden des ausgesendeten Signals in Abhängigkeit der Laufzeit dargestellt.
Der B-Scan oder auch Brightness-Scan ist ein zweidimensionales Bild der Echoamplituden in Helligkeitsabstufungen.
Der TM-Scan oder auch Time-Motion-Scan kann eine zeitliche Bildfolge aufnehmen, durch das auch Bewegungen eine Organes sichtbar gemacht werden können.
