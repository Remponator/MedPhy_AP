\section{Auswertung}
\subsection{Wärmekapazität des Kalorimeters}
Zur Berechnung der Wärmekapazität $c_\text{g}m_\text{g}$ des Kalorimeters wird Gleichung (1) zu
\begin{equation*}
	c_\text{g}m_\text{g}=\frac{c_\text{w}m_\text{h}(T_\text{h}-T_\text{M})-c_\text{w}m_\text{k}(T_\text{M}-T_\text{k}}{T_\text{M}-T_\text{k})}
\end{equation*}
umgeformt und folgende Messwerte verwendet:
\begin{table} [H]
	\centering
	\caption{Messdaten für Bestimmung der Wärmekapazität des Kalorimeters.}
	\label{tab:1}
	\sisetup{table-format=4.2}
	\begin{tabular}{S[table-format=4.2]SSSS}
		\toprule
		{$T_\text{kalt}/\text{K}$}&{$T_\text{heiss}/\text{K}$}&{$T_\text{Misch}/\text{K}$}&{$m_\text{kalt}/\text{g}$}&{$m_\text{heiss}/\text{g}$} \\
		\midrule
		260,1&337,5&286,7&344,1&237,7\\
		262,9&337,5&287,9&296,1&179,6\\
		262,8&337,5&292,2&267,4&198,9\\
		\bottomrule 
	\end{tabular}
\end{table} 
Die spezifische Wärmekapazität $c_\text{w}$ ist aus [1] bekannt und lautet
\begin{equation*}
	c_\text{w} = \SI{4,18}{\joule\per\gram\kelvin}.
\end{equation*}
So ergibt sich
\begin{equation*}
	c_\text{g}m_\text{g} = \SI{290 \pm 90}{\joule\per\kelvin}.
\end{equation*}
Der Mittelwert wurde hierbei anhand der Formel 
\begin{equation} \label{Mittel}
	\overline{x} = \frac{1}{n} \sum_{i=1}^n x_i
\end{equation}
bestimmt. Die Abweichung $\sigma$ mit $i = 1,...,n$
\begin{equation}\label{Abw}
	\sigma_i = \frac{s_i}{\sqrt{n}} = \sqrt{\frac{\sum_{j=1}^n (v_j - \overline{v_i})^2}{n*(n-1)}}.
\end{equation}	

\subsection{Spezifische Wärmekapazität von Aluminium und Blei}
Die Berechnung der Wärmekapazitäten $c_\text{K}$ von Aluminium und Blei erfolgen anhand der Gleichung (1).
Es ergibt sich die Gleichung
\begin{equation}\label{ck}
	c_\text{k}=\frac{(c_\text{W}m_\text{W}+c_\text{g}m_\text{g})(T_\text{M}-T_\text{W})}{m_\text{P}(T_\text{P}-T_\text{M})}
\end{equation}
Bei den Messungen ergeben sich für Aluminium folgende Werte:
\begin{table} [H]
	\centering
	\caption{Messdaten für Bestimmung der Wärmekapazität von Aluminium.}
	\label{tab:2}
	\sisetup{table-format=4.2}
	\begin{tabular}{S[table-format=4.2]SSSS}
		\toprule
		{$T_\text{Wasser}/\text{K}$}&{$m_\text{Wasser}/g$}&{$T_\text{Probe,A}/\text{K}$}&{$m_\text{Probe,A}/\text{g}$}&{$T_\text{Misch}/\text{K}$} \\
		\midrule
		262,1&659,2&373,5&155,0&301,2\\
		260,1&638,4&373,5&155,0&299,6\\
		259,6&617,0&373,5&155,0&299,4\\
		\bottomrule 
	\end{tabular}
\end{table} 
Somit ergeben sich drei verschiedene Wärmekapazitäten:
\begin{gather*}
	c_\text{K,1} = \SI{0,6899 \pm 0,0199}{\joule\per\gram\kelvin} \\
	c_\text{K,2} = \SI{0,6652 \pm 0,0197}{\joule\per\gram\kelvin} \\
	c_\text{K,3} = \SI{0,6472 \pm 0,0198}{\joule\per\gram\kelvin} \\
\end{gather*}
Mit den Gleichungen \ref{Mittel} und \ref{Abw} ergibt sich dann für die Wärmekapazität von Aluminium der ungefähre Wert
\begin{equation*}
	c_\text{K} = \SI{0,667 \pm 0,02}{\joule\per\gram\kelvin}.
\end{equation*}

Zur Berechnung der Wärmekapazität von Aluminium wird analog mit folgenden Messwerten verfahren.
\begin{table} [H]
	\centering
	\caption{Messdaten für Bestimmung der Wärmekapazität von Blei.}
	\label{tab:3}
	\sisetup{table-format=4.2}
	\begin{tabular}{S[table-format=4.2]SSSS}
		\toprule
		{$T_\text{Wasser}/\text{K}$}&{$m_\text{Wasser}/g$}&{$T_\text{Probe,B}/\text{K}$}&{$m_\text{Probe,B}/\text{g}$}&{$T_\text{Misch}/\text{K}$} \\
		\midrule
		260,4&615,7&373,5&542,7&298,5\\
		260,1&628,53&373,5&542,7&298,1\\
		259,5&627,27&373,5&542,7&297,9\\
		\bottomrule 
	\end{tabular}
\end{table} 
Es ergeben sich analog die drei verschiedenen Wärmekapazitäten:
\begin{gather*}
	c_\text{K,1} = \SI{0,1780 \pm 0,0055}{\joule\per\gram\kelvin} \\
	c_\text{K,2} = \SI{0,1802 \pm 0,0054}{\joule\per\gram\kelvin} \\
	c_\text{K,3} = \SI{0,1808 \pm 0,0054}{\joule\per\gram\kelvin} \\
\end{gather*}
Und somit der gemittelte Wert
\begin{equation*}
	c_\text{K} = \SI{0,180 \pm 0,005}{\joule\per\gram\kelvin}.
\end{equation*}

\subsection{Molwärme von Aluminium und Blei}
Die Molwärme $C$ von Aluminium und Blei kann mithilfe von Gleichung (3) berechnet werden.
Es ergibt sich die Formel
\begin{equation}
	C_\text{V} = c_\text{K}\cdot M-9\alpha^2\kappa\frac{M}{\rho}T_\text{M}
\end{equation}
Die Werte M, $\alpha$, $\kappa$ und $\rho$ werden für den jeweiligen Wert der Tabelle aus Abbildung \ref{fig:Konst} entnommen.
\begin{figure}[H]
    \centering
    \caption{Konstanten zur Bestimmung der Molwärme [1].}
    \includegraphics[height=6cm]{Auswertung/Konst.pdf}
    \label{fig:Konst}
\end{figure}
Die restlichen Werte werden den vorherigen Rechnungen entnommen. 
Somit ergeben sich die folgenden Werte für die Molarwärme von Aluminium:
\begin{gather*}
	c_\text{v,1} = \SI{186,24 \pm 5,37}{\joule\per\text{mol}\kelvin}\\
	c_\text{v,2} = \SI{179,55 \pm 5,33}{\joule\per\text{mol}\kelvin}\\
	c_\text{v,3} = \SI{174,69 \pm 5,34}{\joule\per\text{mol}\kelvin}\\
\end{gather*}
Nach den Gleichungen \ref{Mittel} und \ref{Abw} ergibt sich dann ungefähr
\begin{equation*}
	c_\text{v,Aluminium} = \SI{18,016 \pm 0,534}{\joule\per\text{mol}\kelvin}.
\end{equation*}
Analog folgen für Blei die Werte
\begin{gather*}
	c_\text{v,1} = \SI{36,846 \pm 1,129}{\joule\per\text{mol}\kelvin}\\
	c_\text{v,2} = \SI{37,337 \pm 1,123}{\joule\per\text{mol}\kelvin}\\
	c_\text{v,3} = \SI{37,464 \pm 1,129}{\joule\per\text{mol}\kelvin}\\
\end{gather*}
Und somit der ungefähre Wert
\begin{equation*}
	c_\text{v,Blei} = \SI{37,216 \pm 1,127}{\joule\per\text{mol}\kelvin}.
\end{equation*}