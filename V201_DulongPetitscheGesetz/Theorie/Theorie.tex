\section{Zielsetzung}
Das Ziel dieses Versuchs ist es, das Dulong-Petitsche Gesetz auf seine Richtigkeit im Bezug auf die klassische Physik und die Quantenmechanik überprüft.

\section{Theorie}
\subsection{Klassiche Theorie}
Zur Einführung in die klassische Theorie des Dulong-Petitschen Gesetzes wird die Wärmemenge $\Delta Q$, die ein Körper aufnimmt, wenn sich seine Temperatur um $\Delta T$ verändert ohne, dass an ihm Arbeit verrichtet wird, gebildet.
\begin{equation}
    \Delta Q = m \cdot c \cdot \Delta T
\end{equation}
Außerdem wird die spezifische Wärmekapazität $c$ definiert, die auf die Masseneinheit bezogen ist.
Diese ist von der Beschaffenheit des Materials abhängig und hat eine Dimension von $\frac{J}{kg \cdot K}$.
Eine weitere wichtige Größe ist die Molwärme $C$, die die benötige Wärmemenge $dQ$ angibt um ein Mol eines Stoffes um $dT$ zu erwärmen.
Dabei wird zwischen einerseits der Molwärme bei konstantem Volumen $C_V$ und andererseits bei konstantem Druck $C_p$ unterschieden.
\begin{equation}
    C_V = \left (\frac{dU}{dT} \right )_V
\end{equation}
Der Zusammenhang der Molwärme bei konstantem Druck bzw. konstantem Volumen wird folgendermaßen beschrieben.
\begin{equation}
    C_V \cdot C_p = 9\; \alpha^{2}\; \kappa \;V_0 \;T
\end{equation}
Hierbei beschreibt $\alpha$ den linearen Ausdehnungskoeffizienten, $\kappa$ das Kompressionsmodul und $V_0$ das Molvolumen.
Das Dulong-Petitsche Gesetz besagt, dass der Betrag der Molwärme $C_V$ unabhängig von der chemischen Beschaffenheit des Elements einen Wert von $3R$ aufweist, wobei $R$ die ideale Gaskonstante ist.
Aus der Annahme der Gitterkräfte in einem Festkörper und dem Hookschen Gesetz ergibt sich folgende Differentialgleichung.
\begin{equation}
    m \frac{d^{2}x}{dt^{2}} = -Dx
\end{equation}
Die Gleichung (3) beschreibt eine ungedämpfte harmonische Schwingung besitzt die folgende Lösung.
\begin{equation}
    x(t) = A \; cos \left (\frac{2\pi}{\tau}t \right )
\end{equation}
Dabei beschreibt $\tau$ die Periodendauer und $A$ die Amplitude der Schwingung.
Aus der Integration des Hookschen Gesetzes wird die potentielle Energie $E_\text{pot}$ berechnet.
Um die mittlere potentielle Energie zu bestimmen wird die Lösung der Differentialgleichung in die folgende Gleichung eingesetzt.
\begin{equation}
    \langle	E_\text{pot}\rangle = \frac{1}{\tau} \int_0^\tau \! \frac{1}{2} D \; x^{2}(t) dt
\end{equation}
Bei der kinetischen Energie wird die Trägheitskraft eines Körpers mit der Masse $m$ integriert und wieder über die Periodendauer $\tau$ gemittelt.
\begin{equation}
    \langle E_\text{kin} \rangle = \frac{1}{\tau} \int_0^\tau \! \frac{1}{2} m \left ( \frac{dx(t)}{dt}\right )^{2} dt 
\end{equation}
Hierbei ergeben sich für beide mittlere Energien $E_\text{pot}$ und $E_\text{kin}$ identische Werte.
\begin{equation}
    \langle E_\text{pot} \rangle = \langle E_\text{kin} \rangle = \frac{1}{4} D A^{2}
\end{equation}
Somit ergibt sich für die innere Energie eines Atoms unter der Berücksichtigung des Äquipartitionstheorems, welches die kinetische Energie eines Atoms im thermischen Gleichgewicht pro Freiheitsgrad auf $\frac{1}{2}kT$ bestimmen lässt, Folgendes.
\begin{equation}
    \langle u_\text{kl} \rangle = \langle E_\text{pot} \rangle + \langle E_\text{kin} \rangle = 2 \langle E_\text{kin} \rangle = kT
\end{equation}
Die mittlere Energie eines Mols, also $N_L$ Atome, wobei $N_L$ die Loschmidtsche Zahl beschreibt, wird aufgrund der drei Freiheitsgrade jedes Atoms für Festkörper folgendermaßen berechnet.
\begin{equation}
    \langle U_\text{kl} \rangle = 3 \cdot N_L \cdot \langle u \rangle = 3 N_L k T = 3 R T
\end{equation}
Die Molwärme bei konstantem Volumen hat daher erwartungsgemäß einen theoretischen Wert von $C_V = 3R$.

\subsection{Abweichung zur klassischen Physik}
Der Widerspruch der Quantenmechanik besagt im Gegensatz zur klassichen Physik, dass ein harmonischer Oszillator, der mit einer Frequenz $\omega$ schwingt, nur diskrete Energiewerte abgeben bzw. aufnehmen.
Diese Energien haben einen Betrag von $\Delta u = n \hbar \omega$, wobei $n$ aus den natürlichen Zahlen stammt.
Zur Bestimmung der mittleren Energie muss nun über die diskreten Energiewerte multipliziert mit der Auftreffwahrscheinlichkeit $\psi$ integriert und diese dann summiert werden.
$\psi$ ist dabei durch die Boltzmann-Verteilung festgelegt.
Nach der Integration und der Aufsummierung folgt durch eine geometrische Reihe mit $exp \left (-\frac{\hbar \omega}{kT} \right )$.
\begin{equation}
    \langle U_\text{qu} \rangle = \frac{3 N_L \hbar \omega}{exp \left (\frac{\hbar \omega}{kT} \right ) -1}
\end{equation}
Für ein Mol des Festkörpers wird aus den drei Freiheitsgraden und der Loschmidtschen Zahl die mittlere innere Energie bestimmt.
Aus der Taylorentwicklung für die Exponentialfunktion ergibt sich, dass sich $\langle U_\text{qu} \rangle$ für $T \rightarrow 0$ Null nähert.
Für $T \rightarrow \infty$ allerdings strebt die innere Energie gegen den klassich berechneten Wert.
\begin{equation}
    \lim_{T\to\infty} \langle U_\text{qu} \rangle \approx \langle U_\text{kl} \rangle = 3RT
\end{equation}
Dabei wird deutlich, dass die klassische Rechnung nur bei $kt \gg \hbar \omega$ aufgeht.
Außerdem erklärt dieses Verhältnis die Näherung, dass bei kleineren Massen von Atomen höhere Energien gebraucht werden um die Näherung zu erfüllen.
Des Weiteren kann das Spektrum der Eigenfrequenzen, die der Kristall modifiziert, vernachlässigt werden, da dies nur bei tiefen Temperaturen Einfluss hat.