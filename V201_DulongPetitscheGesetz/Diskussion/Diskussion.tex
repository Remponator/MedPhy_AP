\section{Diskussion}
Für die Molwärme von Blei und Aluminium ergaben sich die folgenden Werte:
\begin{gather*}
	c_\text{v,Aluminium} = \SI{18,016 \pm 0,534}{\joule\per\text{mol}\kelvin}\\
	c_\text{v,Blei} = \SI{37,216 \pm 1,127}{\joule\per\text{mol}\kelvin}
\end{gather*}
Der Literaturwert, $C_\text{V} = 3R$, beträgt
\begin{gather*}
	C_\text{V} = 3R = \SI{24,943}{\joule\per\text{mol}\kelvin}. [2]
\end{gather*}
Damit weicht die experimentell ermittelte Molwärme von Aluminium um 38,4\% vom Literaturwert ab, die Molwärme von Blei um 49,2\%.
Bei so großen Abweichungen kann nicht von einer Bestätigung des Gesetztes $C_V = 3R$ gesprochen werden.
Zu kurze Messdurchführungen können zu Fehlern führen. Es kann möglich sein, dass die Proben zu früh wieder aus dem Wasser genommen werden und somit kein exakter Temperaturausgleich stattfinden kann. Zudem kann die Abgabe von Wärme an die Umgebungsluft/-Körper zu Abweichungen der Messergebnisse führen. Auch ist es möglich, dass die Proben nicht komplett erhitzt wurden uns somit nur auf ihrer Außenseite die gewünschte Temperatur hatten. Somit geben sie weniger Wärme an das Wasser ab, als wären sie komplett erhitzt. 
Zuletzt reichen drei Messungen nicht aus, um statistische Fehler zu eliminieren.