\section{Aufbau und Durchführung}
Zur Bestimmung der spezifischen Wärmekapazitäten für verschiedene Stoffe wird ein Kaloriemeter benötigt.
Der erste Schritt, der zur späteren Rechnung essentiell ist, besteht aus der experimentellen Bestimmung der Wärmekapazität des Kaloriemteres.

\subsection{Bestimmung der spezifischen Wärmekapazität des Kaloriemeters}
Zunächst wird die Masse des leeren Kaloriemeters $m_g$ bestimmt.
Dann wird dem leeren Kaloriemeter Wasser hinzugegeben, dessen Masse $m_x$ vorher gewogen wird.
Nachdem sich die Wassertemperatur und die Temperatur der Wand des Kaloriemeters angeglichen haben, wird die Temperatur $T_x$ bestimmt.
Daraufhin wird eine auf die Temperatur $T_y$ erhitzte und eine auf $m_y$ gewogene Menge an Wasser in das Kaloriemeter hinzugegeben.
Wenn sich die beiden Temperaturen erneut angeglichen haben, wird die Mischtemperatur $T_m$ gemessen.

\subsection{Bestimmung der spezifischen Wärmekapazität verschiedener Stoffe}
Zuerst wird die Masse der Probe $m_k$ bestimmt.
Im Anschluss wird die Probe in einem Becherglas mit kochendem Wasser erhitzt.
Währenddessen wird eine gewogene Masse Wasser in das Kaloriemeter gegeben.
Nachdem sich die Temperatur wieder angeglichen hat, wird die Temperatur bestimmt.
Anschließend wird die erhitzte Probe in das Kaloriemeter gegeben.
Auch hier wird gewartet, bis sich die Temperaturen der Probe und des Wassers im Kaloriemeter angeglichen haben.
Es wird erneut die Temperatur gemessen und das erwärmte Wasser gewogen.

\begin{figure}
    \centering
    \includegraphics[height=8cm]{Durchführung/Aufbau.pdf}
    \caption{Schematischer Versuchsaufbau}
    \label{fig:Aufbau}
\end{figure}
