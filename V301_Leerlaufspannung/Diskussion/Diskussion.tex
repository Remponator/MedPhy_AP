\newpage
\section{Diskussion}
Die ersten beiden Messungen zur Bestimmung der Leerlaufspannung und des Innenwiderstandes sind sehr genau, da die Fehler sehr gering sind.
Für die relaitven Fehler des Innenwiderstandes ergeben sich $\sim 1\%$ und für die Leerlaufspannung ergeben sich vernachlässigbar kleine Fehler.
Die Messwerte der beiden Messungen sind konsistent linear und haben nur geringe Abweichungen zur Ausgleichsgerade.
Außerdem sind die Werte der Methode mit und ohne Gegenspannung relativ gleich.
Bei der umgesetzten Leistung ergeben sich für die experimentellen Werte durchweg höhere Werte als die theoretisch Errechneten.
Das lässt sich auf systematische Fehler, Ablesefehler und die Güte der Messgeräte zurückführen.
