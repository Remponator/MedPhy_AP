\section{Zielsetzung}
In diesem Versuch soll die Leerlaufspannung und der Innenwiderstand verschiedener Spannungsquellen gemessen werden.

\section{Theorie}
Wird einer Spannungsquelle kein Strom entnommen, so wird von einer Leerlaufspannung $U_0$ an der Ausgangsklemme gesprochen. 
Fließt über einen äußeren Lastwiderstand $R_\text{a}$ ein endlicher Strom $I$, so kann an der Klemme eine Klemmenspannung $U_\text{k}$ gemessen werden, welche unter dem Wert $U_0$ liegt. Erklärt werden kann dies anhand eines Innenwiderstandes $R_\text{i}$ (siehe Abbildung \ref{fig:Skizze}) der Spannungsquelle. Aus dem Kirchhoff'schen Gesetz
\begin{equation}
	U_0 = I R_\text{i} + I R_\text{a} \notag
\end{equation}
folgt somit für die Klemmenspannung
\begin{equation}
	U_\text{k} = I R_\text{a} = U_0 - I R_\text{i}
\end{equation}
Wird $R_\text{a}$ nun groß gewählt, wird der Strom $I$ klein und es kann die Vereinfachung 
\begin{equation}
	U_\text{k} \approx U_0
\end{equation}
verwendet werden.
\begin{figure}[h]
  \centering
  \includegraphics[height=5cm]{Grafiken/Skizze.pdf}
  \caption{Ersatzschaltbild einer realen Spannungsquelle mit Lastwiderstand $R_\text{a}$, \cite{1}.}
  \label{fig:Skizze}
\end{figure}
Aufgrund des Innenwiderstandes $R_\text{i}$ kann der idealen Spannungsquelle nicht unendlich viel Leistung entnommen werden. 
Die ideale Spannungsquelle hat hierbei die Eigenschaft, eine unabhängige Spannung $U_0$, ohne Innenwiderstand $R_\text{i}$, zu liefern.
Die an $R_\text{a}$ abgegebene Leistung kann nun durch 
\begin{equation}
	N = I^2R_\text{a}
\end{equation}
berechnet werden.
Dabei durchläuft die Leistung N ein Maximum, welches bei $R_i$ erreicht wird. Es handelt es sich um eine Leistungsanpassung. 
Wird der Belastungsstrom verändert, so verändert sich auch das elektrische Verhalten der Quelle. Der Innenwiderstand wird hierbei als eine differentielle Größe betrachtet:
\begin{equation}
	R_\text{i} = \frac{dU_\text{k}}{dI}
\end{equation}



















