\section{Auswertung}

Die Apparatkonstante $K_\text{kl}$ und die Fallstrecke zwischen den Messmarken werden der Anleitung entnommen. Die Massen und Radien der Kugeln, werden gemessen:
\begin{gather*}
	r_\text{kl} = \SI{7,75e-3}{\metre} \\
	r_\text{gr} = \SI{7,9e-3}{\metre}\\
	m_\text{kl} = \SI{0,00484}{\kilogram}\\
	m_\text{gr} = \SI{0,00536}{\kilogram}\\
	K_\text{kl} = \SI{7,640e-8}{\pascal\metre\cubed\per\kilogram}\\
	l = \SI{0,1}{\metre}.
\end{gather*}
Die Dichten der Kugeln kann über die Formel
\begin{equation}
	\rho = \frac{3\cdot m}{4\cdot \pi r^3}
\end{equation}
berechnet werden.
Somit ergeben sich die Dichten:
\begin{gather*}
	\rho_\text{kl} = \rho = \frac{3\cdot m_\text{kl}}{4\cdot \pi r_\text{kl}^3} = \SI{2482.285}{\kilogram\per\metre\cubed} \\
	\rho_\text{gr} = \rho = \frac{3\cdot m_\text{gr}}{4\cdot \pi r_\text{gr}^3} = \SI{2595.343}{\kilogram\per\metre\cubed}
\end{gather*}
Bei 20$^\circ$C beträgt die Dichte von Wasser [2]:
\begin{equation*}
	\rho_\text{Fl} = \SI{998,2}{\kilogram\per\metre\cubed}
\end{equation*}
Nun wird die Apparatkonstante $K_\text{gr}$ bestimmt.
In Tabelle \ref{tab:1 klein} sind Fallzeiten $t_\text{kl}$ der kleinen Kugel mit $r_1$ bei 22$^\circ$C angegeben, in Tabelle \ref{tab:1 groß} die Fallzeiten $t_\text{gr}$ der grossen Kugel mit $r_2$ bei 22$^\circ$C.
\begin{table} [H]
	\begin{minipage}{0.4\textwidth}
	\centering
	\caption{$t_\text{kl}$ der kleinen Kugel.}
	\label{tab:1 klein}
	\sisetup{table-format=2.2}
	\begin{tabular}{S[table-format=2.2]S}
		\toprule
		{$t/s$} \\
		\midrule
		11,67\\
		11,67\\
		11,76\\
		11,58\\
		11,60\\
		11,50\\
		11,58\\
		11,55\\
		11,57\\
		11,55\\
		\bottomrule 
	\end{tabular}
	\end{minipage}\hfill
	\begin{minipage}{0.4\textwidth}
	\centering
	\caption{$t_\text{gr}$ der grossen Kugel.}
	\label{tab:1 groß}
	\sisetup{table-format=2.2}
	\begin{tabular}{S[table-format=2.2]S}
		\toprule
		{$t/s$} \\
		\midrule
		66,09\\
		65,47\\
		66,23\\
		64,92\\
		65,09\\
		66,27\\
		65,76\\
		65,56\\
		65,83\\
		64,87\\
		\bottomrule 
	\end{tabular}
	\end{minipage}\hfill
\end{table}
Für den Mittelwert von $t_\text{kl}$ ergibt sich:
\begin{equation*}
	\overline{t_\text{kl}} = \SI{11,603 \pm 0,024}{\second}
\end{equation*}
Der Mittelwert wurde hierbei anhand der Formel 
\begin{equation}
	\overline{x} = \frac{1}{n} \sum_{i=1}^n x_i
\end{equation}
bestimmt. Die Abweichung $\sigma$ mit $i = 1,...,n$:
\begin{equation}
	\sigma_i = \frac{s_i}{\sqrt{n}} = \sqrt{\frac{\sum_{j=1}^n (v_j - \overline{v_i})^2}{n*(n-1)}}
\end{equation}	
Zunächst muss der Wert der Viskosität ermittelt werden. Hierzu werden die Werte aus Tabelle \ref{tab:1 klein}, $K_\text{kl}$ und $\rho_\text{Fl}$ in Gleichung (2) eingesetzt, daraus folgt:
\begin{equation*}
	\eta = \SI{1,3156\pm0,0027e-3}{\pascal\second}
\end{equation*}
Der Fehler ergibt sich nach der Fehlerfortpflanzungsformel
\begin{equation}\label{Fehler}
	\increment x_i = \sqrt{\biggl(\frac{\partial f}{\partial k_1} * \sigma_{k_1}\biggr)^2 + \biggl(\frac{\partial f}{\partial k_2} * \sigma k_{k_2}\biggr)^2 + ...}.
\end{equation}
Hier folgt für den Fehler von $\eta$ dann
\begin{equation}\label{Feta}
	\increment \eta = \sqrt{\biggl(K\cdot(\rho_\text{K}-\rho_\text{Fl}) \cdot \sigma_{t}\biggr)^2}.
\end{equation}

Um nun $K_\text{gr}$ zu ermitteln, wird (2) nach diesem Wert umgestellt:
\begin{equation*}
	K_\text{gr} = \frac{\eta}{(\rho_\text{gr}-\rho_\text{Fl})\cdot t_\text{gr}} = \SI{9,543\pm ,024e-9}{\pascal\metre\cubed\per\kilogram}
\end{equation*}
Der Fehler wird nach \ref{Fehler} wie folgt berechnet:
\begin{equation*}
	\increment K_\text{gr} = \sqrt{\biggl(\frac{1}{(\rho_\text{gr}-\rho_\text{Fl})\cdot t_\text{gr}} \cdot \sigma_{\eta}\biggr)^2 + \biggl(-\frac{\eta}{(\rho_\text{gr}-\rho_\text{Fl})\cdot t_\text{gr}^2} \cdot \sigma_{t_\text{gr}}}\biggr)^2
\end{equation*}
\begin{table} [H]
	\centering
	\caption{Messung der Fallzeit der grossen Kugel mit $r_2$ bei verschiedenen Temperaturen.}
	\label{tab:2}
	\sisetup{table-format=3.2}
	\begin{tabular}{S[table-format=3.0] S  S S [table-format=3.3] @{${}\pm{}$}S[table-format=2.3] S [table-format=2.3]}
		\toprule
		{$T/°C$} & {$t/s$} & {$\rho_\text{Fl}/ \SI{}{\kilogram\per\metre\cubed}$} & \multicolumn{2}{c}{$\eta / 10^{-3} \text{Pa s}$}\\
		\midrule
		28&59,80&996,8&0,912&0,0023\\
		28&58,47&996,8&0,892&0,0022\\
		32&54,95&995,0&0,839&0,0022\\
		32&54,13&995,0&0,827&0,0021\\
		36&51,09&993,7&0,781&0,0020\\
		36&50,04&993,7&0,765&0,0019\\
		40&47,40&992,2&0,725&0,0018\\
		40&46,69&992,2&0,714&0,0018\\\
		47&42,76&989,4&0,655&0,0016\\
		47&41,93&989,4&0,643&0,0016\\
		51&39,61&987,6&0,608&0,0015\\
		51&38,67&987,6&0,593&0,0015\\
		56&36,23&985,2&0,557&0,0014\\
		56&36,32&985,2&0,558&0,0014\\
		60&34,20&983,2&0,526&0,0013\\
		60&34,09&983,2&0,524&0,0013\\
		65&32,40&980,6&0,499&0,0013\\
		65&32,10&980,6&0,495&0,0012\\
		70&30,27&977,8&0,467&0,0012\\
		70&29,78&977,8&0,460&0,0012\\
		\bottomrule 
	\end{tabular}
\end{table}
Der Fehler von $\eta$ wird mit Gleichung \ref{Feta} berechnet.
Anhand der Werte aus Tabelle \ref{tab:2} können nun die Konstanten der Andradeschen Gleichung (3) bestimmt werden.
Für die Dichte des Wassers $\rho_\text{Fl}$, bei den verschiedenen Temperaturen, wurden Literaturwerte verwendet [2].Die Viskosität wurde mithilfe der Gleichung (2) bestimmt.
Die Messwertpaare mit selber Temperatur werden jeweils wie oben gemittelt.
\begin{figure}[H]
    \centering
    \includegraphics[width=0.9\textwidth]{Auswertung/Graph1.pdf}
    \caption{Bestimmung der Konstanten der Andradeschen Gleichung.}
	\label{fig:graph}
\end{figure} 
Mithilfe der Pythonfunktion CurveFit können so die beiden Konstanten der Andradeschen Gleichung berechnet werden. Dabei wird die Ausgleichsfunktion der Form $\text{A}\cdot x + \text{B}$ verwendet.
\begin{gather*}
	\text{A} = \SI{8,613\pm0,005e-6}{\pascal\second}\\
	\text{B} = \SI{3164,07\pm905,73}{\kelvin}
\end{gather*}
Zuletzt werden, zur Überprüfung der Laminarität der Strömung, die Reynoldszahlen bestimmt. Durch Gleichung (4) erhält man somit für die beiden Kugeln:
\begin{gather*}
	\text{kleine Kugel: Re} = 331.6 \pm 0.7\\
	\text{grosse Kugel: Re} = 62-5 \pm 0.16
\end{gather*}
Der Fehler ergibt sich nach \ref{Fehler}. Somit folgt 
\begin{equation*}
	\increment Re = \sqrt{\biggl(-\frac{\rho\cdot v\cdot d}{\eta^2} \cdot \sigma_{eta}\biggr)^2}
\end{equation*}