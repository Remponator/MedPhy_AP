
\section{Zielsetzung}
Das Ziel dieses Versuchs ist es, die dynamische Viskosität von destilliertem Wasser zu bestimmen.
Dafür wird die Temperaturabhängigkeit berücksichtigt und untersucht.

\section{Theorie}
Die dynamische Viskosität $\eta$ beschreibt die Eigenschaft einer Flüssigkeit, einen Widerstand für einen sich durch die Flüssigkeit bewegenden Körpers darzustellen.
Diese ist dabei sehr temperaturabhängig und ist von der Geschwindigkeit $v$ und der Berührungsfläche $A$ abhängig.
Zur Bestimmung der Viskosität wird das Kugelfall-Viskosimeter nach Höppler verwendet.
Der oben beschriebene Widerstand wird durch Reibung ausgelöst, die Stokessche Reibung.
Am Beispiel einer Kugel, die im Viskosimeter verwendet wird, ergibt sich folgender Zusammenhang:
\begin{equation}
    F_R = 6  \; \pi \; \eta \; v \; r
\end{equation}.
Zusätzlich zu der Reibungskraft $\vec{F}_R$ wirken die Gewichtskraft $\vec{F}_G$ und die Auftriebskraft $\vec{F}_A$, die der Gewichtskraft entgegenwirkt.
Da die Reibungskraft mit wachsender Geschwindigkeit auch größer wird, stellt sich nach einer bestimmten Zeit ein Kräftegleichgewicht ein, wodurch sich die Kugel mit einer konstanten Geschwindigkeit weiterbewegt.
Die dynamische Viskosität lässt sich dann aus der Dichte der Kugel $\rho_\text{K}$, der Dichte der Flüssigkeit $\rho_\text{Fl}$, der Fallzeit $t$ und der Apparaturkonstante $K$ nach
\begin{equation}
    \eta = K \cdot (\rho_K-\rho_\text{Fl}) \cdot t
\end{equation} 
bestimmen.
Dabei setzt sich $K$ unter Anderem aus der Fallhöhe und der Kugelsymmetrie zusammen.
Zusätzlich lässt sich aufgrund der starken Temperaturabhängigkeit der dynamischen Viskosität die Andradsche Gleichung aufstellen.
\begin{equation}
    \eta \:(T) = A \cdot exp \left (\frac{B}{T} \right )
\end{equation}
Hinzufügend wird die Reynoldszahl für das Fallrohr bestimmt, die angibt ob eine laminare oder turbulente Strömung vorliegt.
Dazu wird folgende Gleichung verwendet.
\begin{equation}
    Re = \frac{\rho \cdot v \cdot d}{\eta}
\end{equation}
Hierbei ist $v$ die Geschwindigkeit der fallenden Kugel, $d$ der Durchmesser der Kugel und $\eta$ die Viskosität der Flüssigkeit.
Dabei wird der Wert $Re_\text{krit} \approx 2300$ definiert, ab dem die Strömung von einer laminaren auf eine turbulente Strömung wechselt.

