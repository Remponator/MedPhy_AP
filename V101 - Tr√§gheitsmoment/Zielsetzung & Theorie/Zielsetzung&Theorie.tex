\documentclass{scrartcl}

\usepackage[aux]{rerunfilecheck}

\usepackage{fontspec}

\usepackage{amsmath}
\usepackage{amssymb}
\usepackage{mathtools}

\usepackage[
	math-style=ISO,
	bold-style=ISO,
	sans-style=italic,
	nabla=upright,
	partial=upright,
]{unicode-math}
\setmathfont{Latin Modern Math}


\usepackage{polyglossia}
\setmainlanguage{german}

%mehr Pakete hier

\usepackage[unicode]{hyperref}
\usepackage{bookmark}
%Einstellung hier, z.B. Fonts

\begin{document}
\section{Zielsetzung}
In dem Versuch ’Das Trägheitsmoment’ soll das Trägheitsmoment verschiedener Geometrien und einer Holzpuppe bestimmt werden und der Satz von Steiner bestätigt werden.

\section{Theorie}
Wird ein drehbarer Körper in Rotations versetzt, so verhält sich dieser träge gegenüber der wirkenden Kraft. Diese Trägheit wird beschrieben durch das Trägheitsmoment $I$. Bei einer punktförmigen Masse ist I definiert als:

\begin{equation}
	I = mr^2
\end{equation}

Bei räumlichen Körper ist somit das Trägheitsmoment

\begin{equation}
	I = \int r^2 dm
\end{equation}

Für manche geometrischen Körper (z.B. Kugel, Zylinder) gibt es bereits definierte Formeln. Bei komplexeren Körpern müssen diese in einfachere, bekannte Körper zerlegt werden und die Trägheitsmomente schließlich aufaddiert werden.

Nicht immer ist der Schwerpunkt in der Drehachse des Körpers. Ist die Schwerpunktsachse des Körpers parallel zur Drehachse verschoben, so kann der $Satz \, von \, Steiner$ angewendet werden:

\begin{equation}
	I = I_s+m*a^2
\end{equation}

$I_s$ ist hierbei das Trägheitsmoment ausgehend von der Schwerpunktachse, $m$ die Gesamtmasse und $a$ der Abstand der beiden Achsen voneinander.

Die auf den Körper wirkende Kraft $\vec{F}$ im Abstand $\vec{r}$ von der Achse wird als Drehmoment $M$ definiert:
\begin{equation}
	\vec{M} = \vec{F} \times \vec{r}
\end{equation}

Kann das System in Schwingung versetzt werden, so wirkt während der Drehung des Körpers (um Winkel $\rho$ aus der Ruhelage), eine rücktreibende Kraft entgegen, im hier vorliegenden Versuch, durch eine Feder hervorgerufen.
Beim Loslassen des Körpers bewirkt dies eine harmonische Schwingung des Körpers. Die Schwingungsdauer wird hierbei beschrieben durch
\begin{equation}
	T = 2\pi\sqrt\frac{I}{D}
\end{equation}

$D$ ist die sog. Winkelrichtgröße. Mithilfe dieser lässt sich nun das Drehmoment berechnen:
\begin{equation}
	M = D*\rho
\end{equation}

\end{document}
