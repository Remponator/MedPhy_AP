\section{Diskussion}
Werden die im Versuch ermittelten Halbwertszeiten mit den theoretischen Werten verglichen, ergeben sich folgende Abweichungen: \\
Indium \cite{2}:
\begin{gather*}
	T_\text{gemessen} = (3412.01 \pm 0.0004) \text{s} = 56.87\pm6.7\cdot10^{-7} \text{min} \\
	T_\text{theoretisch} = 54.3 \text{min} 
\end{gather*}
Für Indium ergibt sich somit nach
\begin{equation}
	\text{Abweichung} = \frac{T_\text{gemessen} - T_\text{theoretisch}}{T_\text{gemessen}} \cdot 100\%
\end{equation}
eine Abweichung von ca. 4.5\%.
\\
Silber, $Ag^{108}$\cite{3}:
\begin{gather*}
\label{eq:Abw}
	T_\text{gemessen} = 130.886 \pm 0.005 \text{s} = 2.18 \text{min} \\
	T_\text{theoretisch} =  2.38 \text{min} 
\end{gather*}
Für $Ag^{108}$ ergibt sich somit nach \ref{eq:Abw} eine Abweichung von ca. 9.2\%.
\\
Silber, $Ag^{110}$\cite{3}:
\begin{gather*}
\label{eq:Abw}
	T_\text{gemessen} = 14.697 \pm 0.022 \text{s} \\
	T_\text{theoretisch} =  24.56 \text{s} 
\end{gather*}
Für $Ag^{110}$ ergibt sich somit nach \ref{eq:Abw} eine Abweichung von ca. 67.1\%.
\\
Auffällig ist die hohe Abweichung bei dem kurzlebigen Silber-Isotop. Es ist möglich, dass bereits in dem Zeitraum des Herausnehmens der Probe bis zum Startpunkt der Messung ein erheblicher Anteil des Isotops Zerfallen ist und somit die Halbwertszeit scheinbar früher erreicht wird. Die Summenkurve \ref{fig:sum} zeigt sehr gut den Verlauf der beiden Zerfälle und ihre unterschiedlichen Halwertszeiten. Die Ungleichung wurde bestätigt. Jedoch entspricht sie nicht der Bedingung, dass $N_{\increment, kurz}(t_i)$ viel kleiner $N_{\increment, lang}(t_i)$ ist, weshalb die Rechnungen teils ungenau sind. Jedoch handelt es sich, nach mehrfacher Variation, bei dem gewählten $t$ um den best-möglichen Wert. Der Zeitpunkt $t^*$ ist somit richtig gewählt.
\\
Die Messung des Indiums hingegen ist sehr genau. Hierbei handelt es sich auch um eine höhere Halbwertszeit, weshalb die Überführung der Probe keinen großen Einfluss auf die Messgenauigkeit hat. 
\\
Die Größe $N(0)$ bei den beiden Silber-Isotopen liegen ungewöhnlich weit auseinander. 
Während bei dem $Ag^{108}$-Isotop der Wert $N(0) = 51.77 \pm 5.51$ beträgt, so liegt er bei dem $Ag^{110}$-Isotop lediglich bei $N(0) = 29.13 \pm 0.21$. Gewöhnlich sind die beiden Werte ungefähr identisch. Dies kann an dem zuvor genannten Problem der sehr kurzen Halbwertszeiten liegen. Durch die verfälschte Ausgleichsrechnung werden hier auch verfälschte Werte errechnet. 