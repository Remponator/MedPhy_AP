\section{Zielsetzung}
Im folgenden Versuch wird anhand der Methode der Aktivierung mit Neutronen die Halbwertszeit von zwei verschiedenen Nukliden bestimmt.

\section{Theorie}
Bei der Aktivierung mit Neutronen handelt es sich um einen Versuch mit radioaktiven Zerfällen.
Dabei zerfallen instabile Atomkerne in stabile Atomkerne. 
Stabile Nuklide zeichnen sich durch ein festes Verhältnis von Neutronen und Protonen aus.
Des Weiteren kann die Zerfallswahrscheinlichkeit der instabilen Nuklide durch die Halbwertszeit ausgedrückt werden, die angibt in welcher Zeitspanne von einer bestimmten Menge instabiler Nuklide noch die Hälfte vorhanden ist.

\subsection{Kernreaktion mit Neutronen}
Zur Erzeugung von instabilen Nukliden werden stabile Nuklide mit Neutronen beschossen.
Dabei müssen die Neutronen das Coulomb-Potential des Kerns nicht überwinden und brauchen somit weniger Energie.
 
Grundlegend wird also ein Kern A durch Beschuss eines Neutrons zum Zerfall angeregt.
Zunächst fusionieren dabei der Ausgangskern A mit einem Neutron, wodurch der Kern in einen höheren Energiezustand A*, der auch als Zwischenkern bezeichnet wird, wechselt.
Um in den Grundzustand zu wechseln wird Energie in Form eines $\gamma$-Quants frei.
Die Reaktion
\begin{equation}
    \label{eq:1}
    _{z}^{m} \text{A} + _{0}^{1} \text{n} \rightarrow _{z}^{m+1} \text{A*} \rightarrow _{z}^{m+1} \text{A} + \gamma \notag
\end{equation}
bringt aber einen meist instabilen Kern hervor, da das Verhältnis zwischen der Kernladungszahl und der Ordnungszahl verändert wurde.

Dabei ist der entstandene Kern langlebiger als der Zwischenkern, wandelt sich aber aufgrund seiner Instabilität durch Emission eines Elektrons in einen stabilen Kern nach
\begin{equation}
    \label{eq:2}
    _{z}^{m+1} \text{A} \rightarrow _{z+1}^{m+1} \text{B} + \beta^{-} + E_\text{kin} + \overline{\nu}_e 
\end{equation}
um. Dabei ist $\overline{\nu}_e$ ein Antineutrino und die kinetische Energien des Elektrons und des Antineutrinos $E_\text{kin}$ der Massendefekt
\begin{equation}
  \Delta E = \Delta m c^2, \notag
\end{equation}
der sich aus der Massendifferenz $\Delta m$ der beiden Kerne A und B ergibt.

\subsection{Schnelle und langsame Neutronen}
Zusätzlich muss die Wahrscheinlichkeit für den Einfang eines Neutrons durch einen stabilen Kern berücksichtig werden.
Dabei wird der sogenannte Wirkungsquerschnitt $\sigma$ definiert, der die Fläche beschreibt, die der Kern haben müsste um jedes auftreffende Neutron eingefangen werden würde.
Es gilt bei einem Beschuss von $n$ Neutronen pro Sekunde und $u$ Einfängen auf eine $\SI{1}{cm^2}$ Fläche einer dünnen Folie 
\begin{equation}
    \sigma = \frac{u}{n K d},    \notag
\end{equation} 
wobei $d$ die Dicke der Folie und $K$ die Anzahl der Atome pro Volumen angibt.
Der Wirkungsquerschnitt wird in barn angegeben wobei $1\text{barn} = 10^{-24}\text{cm}^2$ ist.
Dabei ist $\sigma$ stark geschwindigkeitsabhängig.
Dazu wird mittels der De-Broglie-Wellenlänge
\begin{equation}
    \lambda = \frac{h}{m_n v}   \notag
\end{equation}
zwischen langsamen und schnellen Neutronen differenziert.
Dabei gilt bei hohen Geschwindigkeiten, also bei kleinen Wellenlängen gegen den Kernradius, dass es über geometrische Überlegungen zur Streuung kommt.
Bei niedrigen Geschwindigkeiten hingegen ist aufgrund des Verhältnisses von Wellenlänge zum Kernradius der Verlgleich zur geometrischen Optik nicht anwendbar.
Es treten bei bestimmten Neutronengeschwindigkeiten Maxima des Wirkungsquerschnitts auf, weil es zur Resonanzabsorption kommt.
Die Energie der Neutronen entspricht dann der Differenz zweier Energieniveaus des Zwischenkerns.
Der Wirkungsquerschnitt wird nach Breit und Wigner dann nach
\begin{equation}
    \sigma (E) = \sigma_0 \sqrt{\frac{E_\text{r}}{E}} \frac{\tilde{c}}{(E-E_\text{r})^2 + \tilde{c}}    \notag
\end{equation}
bestimmt, wobei $\tilde{c}$ und $\sigma_0$ charakteristische Grö\ss{}en für die ablaufende Kernreaktion sind.
Falls die Energie des Neutrons gleich der Energie des Niveaus ist, wird der Term maximal.
Mit der weiteren Annahme $E<<E_r$ ergeben sich die reziproken Proportionalitäten
\begin{equation}
    \sigma \propto \frac{1}{\sqrt{E}} \propto \frac{1}{v}.
\end{equation}
Somit werden die langsamen Neutronen mit einer höheren Wahrscheinlichkeit absorbiert als schnelle Neutronen.

\subsection{Erzeugung langsamer Neutronen}
Da Neutronen aufgrund ihrer Instabilität nicht in der Natur vorkommen, werden sie durch den Beschuss von $\alpha$-Strahlung auf $^{9} \text{Be}$ erzeugt.
Dabei stammt die $\alpha$-Strahlung aus dem Zerfall von $^{226} \text{Ra}$.
Somit folgt die Reaktionsgleichung
\begin{equation}
    _{4}^{9} \text{Be} + _{2}^{4} \text{\alpha} \rightarrow _{6}^{12} \text{C} _{0}^{1} \text{n},    \notag
\end{equation}
wobei die hier entstehenden Neutronen ein breites kontinuierliches Energiespektrum bis zu $\SI{13,7}{Mev}$ aufweisen.
Die Neutronen werden durch dicke Materieschichten geleitet, wodurch elastische Stö\ss{}e dann zur Abbremsung führen.
Dabei ist die Wirkung der Verlangsamung maximal, wenn sich die Massen der beiden Sto\ss{}partner, also die der Neutronen und der Atome der Materieschicht, nicht unterscheiden.
Am besten geeignet dafür ist Wasserstoff.
Somit ergibt sich durch mehrfache Stö\ss{} eine mittlere kinetische Energie von $\SI{0,025}{eV}$ bei einer Temperatur von $\SI{290}{K}$.
Das ergibt eine mittlere Geschwindigkeit für die sogenannten thermischen Neutronen von $\SI{2,2}{\frac{km}{s}}$.

\subsection{Der Zerfall instabiler Isotope}
Die Proben werden nach Gleichung \ref{eq:1} zunächst durch den Beschuss mit Neutronen zu einem instabilen Atom angeregt.
Dann zerfällt der instabile Kern nach Gleichung \ref{eq:2} durch die $\beta^{-}$-Strahlung in einen stabilen Zustand.
Für Indium ergibt sich
\begin{equation}
    _{49}^{115} \text{In} + _{0}^{1} \text{n} \rightarrow _{116}^{49} \text{In} \rightarrow _{116}^{50} \text{Sn} + \beta^{-} + \overline{\nu}_e.   \notag
\end{equation}
Für Silber ergeben sich aufgrund der Tatsache, dass natürliches Silber aus zwei Isotopen besteht die Reaktionsgleichungen
\begin{equation}
    _{47}^{107} \text{Ag} + _{0}^{1} \text{n} \rightarrow _{47}^{108} \text{Ag} \rightarrow _{48}^{108} \text{Cd} + \beta^{-} + \overline{\nu}_e  \notag
\end{equation}
und 
\begin{equation}
    _{47}^{109} \text{Ag} + _{0}^{1} \text{n} \rightarrow _{47}^{110} \text{Ag} \rightarrow _{48}^{110} \text{Cd} + \beta^{-} + \overline{\nu}_e.  \notag
\end{equation}

Nach dem allgemeinen Zerfallsgesetz
\begin{equation}
    N(t) = N_0 \text{e}^{-\lambda t},   \notag
\end{equation}
wobei $\lambda$ die Zerfallskonstante und $N_0$ die Anzahl der instabilen Kerne ist.
Die Halbwertszeit $T$ beschreibt die Zeit, nach der die Hälfte der instabilen Kerne zerfallen ist.
Nach
\begin{equation}
    \label{eq:3}
    \frac{1}{2} N_0 = N_0 \text{e}^{-\lambda T} \leftrightarrow T = \text{ln}\left(\frac{2}{\lambda}\right)
\end{equation}
lässt sich die Halbwertszeit bestimmen.
Praktisch lässt sich die Halbwertszeit aber besser durch die Betrachtung von $N(t)$ in einem Zeitintervall $\Delta t$ bestimmen.
Dann ergibt sich zunächst
\begin{equation}
    N_{\Delta t} (t) = N(t) - N(t + \Delta t)  \notag
\end{equation}
woraus sich durch Umformungen die Zerfallskonstante $\lambda$ mittels lineare Regression nach
\begin{equation}
\label{eq:4}
    \text{ln}(N_{\Delta_t}) = \text{ln}(N_0 (1 - \text{e}^{-\lambda \Delta t})) - \lambda t.
\end{equation}
Dabei ist $\text{ln}(N_0 (1 - \text{e}^{-\lambda \Delta t}))$ konstant und $\lambda$ gibt die Steigung die Geraden an.
Es sollte darauf geachtet werden, dass $\Delta t$ in einem gewissen Rahmen liegt, da es bei zu kleinen Zeitintervallen zu statistischen und bei zu gro\ss{}en Zeitintervallen zu systematischen Fehlern kommt.
