\section{Aufbau \& Durchführung}
Der prinzipielle Aufbau der Messvorrichtung ist in Abbildung \ref{fig:aufbau} dargestellt.
\begin{figure}[h]
    \centering
    \includegraphics[height=4cm]{Durchführung/Aufbau.pdf}
    \caption{Prinzipieller Aufbau der Messapparatur \cite{V702}.}
    \label{fig:aufbau}
\end{figure}
Über ein Geiger-Müller-Zählrohr werden die $\beta$- und $\gamma$-Teilchen als elektrische Impulse aufgezeichnet.
Dabei wird es zur Verminderung der äu\ss{}eren Strahlungseinflüsse von Blei umgeben.
Über den Verstärker lassen sich die Zählraten auf einem bestimmten Zeitintervall $\Delta t$ messen.
Es werden zwei Zähler verwendet, da somit kein Zeitverlust durch das Zurücksetzen der Zählrate und durch das Neustarten des Zählrohrs entsteht.

Des Weiteren sollte der Nulleffekt in der Auswertung berücksichtigt werden.
Dabei wird eine Messung über $\SI{900}{s}$ ohne Probe durchgeführt und von allen Messungen im Folgenden abgezogen.

Dann werden für die beiden Proben geeignete Zeitintervalle aus den vorgegebenen Bereichen gewählt, die Proben in die Vorrichtung eingesetzt und eine vorgegebene Anzahl von Messwerten aufgenommen.
Für Indium werden $20$ Werte jeweils mit einem Zeitintervall von $\SI{225}{s}$ aufgenommen.
Bei Silber werden $50$ Werte mit einem deutlich kleineren Zeitinvervall von $\SI{9}{s}$ aufgenommen.
Dabei sollte darauf geachtet werden, dass die Messwerte schnell notiert werden.
