\section{Durchführung}

Die Durchführungen der Versuche zur Bestimmung des Schubmoduls und des magnetischen Momentes sind nahezu identisch.
Zu Beginn wird der reflektierte Lichtstrahl durch das Justierrad so eingestellt, dass dieser einen geringen Abstand zum Lichtdetektor aufweist.
Um an dem Draht Torsionsschwingungen zu erzeugen, wird das Justierrad einmal um einen kleinen Winkel aus der eingestellten Ausgangslage heraus- und wieder zurückgedreht.
Dabei sollte beachtet werden, dass die Kugel keine Pendelbewegung ausführt, sondern nur rotiert.
Das Element des Aufbaus aus Abbildung 5 misst die Schwingungsdauer, da sich der reflektierte Lichtstrahl durch die Torsion am Lichtdetektor vorbei bewegt und somit anfängt die Zeit zu messen.
Bei dem nächsten Überlaufen des Lichtstrahls über den Detektor, wird die Zeit noch nicht gestoppt, sondern erst nachdem der Lichtstrahl erneut den Detektor übertreten hat.
Außerdem sollte bei der Bestimmung des Schubmoduls die Dipolachse des in der Kugel enthaltenden Magneten parallel zu dem Torsionsdraht stehen, damit das Magnetfeld der Erde die Messung nicht beeinflusst.
Bei der Bestimmung des magnetischen Momentes wird zusätzlich die Stromquelle für das Helmholtzspulenpaar eingeschaltet, sodass ein Magnetfeld erzeugt wird.
Außerdem muss darauf geachtet werden, dass die Dipolachse parallel zu dem magnetischen Feld ausgerichtet ist um Messungenauigkeiten zu vermeiden.
Die Dipolachse ist durch einen weißen Punkt auf der Kugel gekennzeichnet.
