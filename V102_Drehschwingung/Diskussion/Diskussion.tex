\section{Diskussion}

Bei der Durchführung des Versuches 'Drehschwingungen' konnte es durch mehrere Fehlerquellen zu Ungenauigkeiten bei der Messung und Auswertung kommen. Obwohl statt Scherung die Torsion verwendet wird, wird die elastische Nachwirkung zwar minimiert aber nicht komplett vermieden. Zudem war es nicht möglich die Schwingung bei sehr kleinem Auslenkwinkel zu messen was dazu führt, dass die Kleinwinkelnäherung noch ungenauer ist. Pendelbewegungen der Kugel konnten nicht komplett vermieden werden, was zur Beeinflussung der Schwingung und somit der genauen Messung führt. 

Die Durchführung des Versuches ergab folgende Werte:
\begin{quote} 
$G = \SI{1.678\pm0.1018}{\num{e11}\newton\per\square\metre}$  \\
$\mu = -0,37 \pm 0,04$ \\
$E = \SI{21.00\pm0.05}{\num{e10}\newton\per\square\metre}$ \\
$Q = \SI{4.00\pm0.17}{\num{e10}\pascal}$ \\
$m_\text{magn} = \SI{0.05601\pm0.00028}{\ampere\square\metre}$
\end{quote} 

Alle berechneten Größen sind durch die vielen einfließenden Messewerte stark Fehler durchzogen.
Das magnetische Moment ist jedoch trotz alledem mit einem kleinen Fehler sehr genau.

Verglichen mit den Literaturwerten für Schubmodule verschiedener Stoffe, kann angenommen werden, dass es sich bei dem hier verwendeten Stoff um Kupfer ($G = \SI{125}{\num{e9}\newton\per\square\metre}$) oder Stahl ($G = \SI{200}{\num{e9}\newton\per\square\metre}$) handelt [2].