\section{Diskussion}
\label{Diskussion}
Die bestimmten Kennlinien und Werte liegen alle nah an den theoretisch erwarteten Messwerten.
Die Fits passen sehr gut zu den Theoriewerten und decken sich sehr gut. 
Der Sättigungsstrom werden bei fast allen Messreihen erreicht.
\begin{gather*}
	I_\text{S1} = 0,162 \text{mA} \\
	I_\text{S2} = 0,368 \text{mA} \\
	I_\text{S3} = 0,785 \text{mA} \\
	I_\text{S4} = 1,353 \text{mA} \\
	I_\text{S5} = 3,000 \text{mA}
\end{gather*}
Lediglich bei der maximalen Heizleistung kann kein Sättigungsbereich gefunden werden. Bei den anderen Messreihen konnte $I_\text{S}$ jedoch leicht abgelesen / bestimmt werden. \*
Auch die Langmuir-Schottkysche Raumladungsgleichung kann bestätigt werden, mit einem Wert von $q = \num{1.4163 \pm 6.6587}{10^{-5}}$,, bei einem Literaturwert von $q_{\text{theo}} = \frac{3}{2} \qquad [1]$ (Abweichung von 5,9\%).
Es besteht nur eine sehr geringe Abweichung bei den gemessenen Werten zum Theoriewert. \\*
Die durch Regression bestimmte Kathodentemperatur (T = 2411.1 K) weicht jedoch stark von den, über eine Leistungsbilanz, bestimmten Temperaturen ab (1399 - 1664 K). Beachtenswert ist hierbei der große Abstand der beiden berechneten Temperaturen, obwohl sie bei den gleichen Heizleistungen bestimmt wurden.\*
Für die Austrittsarbeit $W_\text{A}$ von Wolfram ergibt sich $W_\text{A} = \num{4.36 \pm 0.07} \text{eV}$. Der theoretische Wert beträgt $W_\text{A} \approx 4,6 \text{eV}  [2]$, was zu einer Abweichung von 5,5\% führt.