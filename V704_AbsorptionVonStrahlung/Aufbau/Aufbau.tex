\section{Aufbau \& Durchführung}
Der schematische Aufbau ist in Abbildung \ref{fig:Aufbau} zu sehen.
\begin{figure}[H]
  \centering
  \includegraphics[height=3.8cm]{Aufbau/Aufbau.pdf}
  \caption{Schematischer Aufbau zur Messung von $\gamma$- und $\beta$-Strahlung.}
  \label{fig:Aufbau}
\end{figure}
Die Strahlung aus der $\gamma$- bzw. $\beta$-Quelle trifft auf eine variierbare Absorberplatte. Das dahinter liegende Geiger-Müller-Zählrohr detektiert die restliche, nicht abgeschirmte Strahlung.
\\
Bei der Messung der $\beta$-Strahlung wird lediglich keine Blei-Abschirmung verwendet, da diese schlichtweg nicht benötigt wird.
Um später die Hintergrundmessung von den Messwerten differenzieren zu können, wird über 900 s die Nullmessung durchgeführt. Dabei wird nur die Hintergrundstrahlung aufgenommen. 

\subsection{Messung der Gamma-Absorptionskurve}
Nun wird die $\gamma$-Absorptionskurve von $^{137}$$\text{Cs}$ aufgenommen. 
Zwischen dem Zählrohr und der Probe werden verschiedene Abschirmungsmaterialien angebracht. Begonnen wird mit zehn Bleiplatten verschiedener Dicke, wobei die Messzeit für höhere Dicken auch erhöht wird. Der Fehler der Messwerte beträgr $\frac{\sqrt{N}}{N}$. Bei den Messreihen muss darauf geachtet werden, dass dieser statistische Fehler nicht über 3\% steigt. Danach werden die Messungen für Zink als Absorbermaterial wiederholt. 

\subsection{Messung der Beta-Absorptionskurve}
Im zweiten Versuchsteil wird die Absorptionskurve des $\beta$-Strahlers $^{99}$$\text{Tc}$ aufgenommen. 
Hierzu werden nacheinander zehn verschiedene Aluminiumplatten mit verschiedenen Dicken zwischen den Strahler und das Zählrohr geklemmt. Bei der geringsten Dichte wird eine Messdauer von 100 s gewählt. Bei jeder folgenden Messung wird die Messdauer um 100 s erhöht, bis die letzte Messung über eine Messzeit von 1000 s durchgeführt wird.