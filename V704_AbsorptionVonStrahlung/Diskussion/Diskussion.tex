\section{Diskussion}
\subsection{Gamma-Strahlung}
Bei der $\gamma$-Strahlung ergaben sich für Blei und Zink sehr unterschiedliche Abweichungen.
Für Zink ergaben sich der empirisch ermittelte Wert $\mu_\text{exp} = \SI{47.4 \pm 1.9}{\frac{1}{m}}$ und für den Compton-Absoprtionskoeffizienten $\mu_\text{Comp} = \SI{50.380}{\frac{1}{m}}$.
Bei der Messung mit Blei ergaben sich $\mu_\text{exp} = \SI{102.1 \pm 1.6}{\frac{1}{m}}$ und $\mu_\text{Comp} = \SI{69.229}{\frac{1}{m}}$
Der experimentell ermittelte Absorptionskoeffizient von Blei hat zu dem Compton-Absorptionskoeffizienten nach
\begin{equation}
  f = 1 - \frac{\mu_\text{exp}}{\mu_\text{Comp}}  \notag
\end{equation}
eine Abweichung von $47,5\%$, wobei $f$ die relative Abweichung ist.
Die Messung mit Zink ergab dagegen eine Abweichung von nur $6 \%$.  \\
Trotz der großen Abweichungen bei der Messung mit Blei kann auf eine zuverlässige Messung geschlossen werden, da die beiden Anfangsaktivitäten sehr nahe beieinander liegen, trotz der Tatsche dass diese Werte nach der Regression noch exponiert werden mussten.
Dabei beträgt die Anfangsaktivität für Zink $A(0)_\text{Zink} = \SI{137.414 \pm 1.025}{\frac{1}{s}}$ und für die Bleiabschirmung $A(0)_\text{Blei} = \SI{130.321 \pm 1.039}{\frac{1}{s}}$.
Die auftretende Abweichung $1 - \frac{A(0)_\text{Zink}}{A(0)_\text{Blei}}$ beträgt $ 5\%$ und ist somit mit Berücksichtigung auf statistische Messfehler gering. \\
Die hohe Abweichung des Compton-Absorptionskoeffizienten zum experimentell ermittelten Wert lässt darauf schließen dass noch ein anderer Effekt, der Photoeffekt, neben dem Compton-Effekt während der Messung mit Blei eingetreten ist.
\subsection{Beta-Strahlung}
Die experimentell bestimmte maximale Energie beträgt $E_\text{max} = \SI{0.234 \pm 0.149}{MeV}$.
Bei der Untersuchung der $\beta$-Strahlung fällt auf, dass der Fehler der maximalen Energie $\frac{\Delta E_\text{max}}{E_\text{max}}$ eine Abweichung von $57\%$ beträgt.
Das kann aus der Tatsache entsprungen sein, dass das Geiger-Müller-Zählrohr während des Experimentes aufgrund einer Disfunktion neu gestartet werden musste.
Des Weiteren wurde gezeigt, dass $\beta$-Strahlung bei einer Abschirmung durch Aluminium eine maximale Reichweite von $r_\text{max} = \SI{20}{cm}$ erreicht.
