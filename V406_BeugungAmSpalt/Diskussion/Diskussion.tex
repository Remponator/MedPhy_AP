\section{Diskussion}

Auffällig bei den Messwerten ist, dass sie teilweise nicht gleich auf beiden Seiten verlaufen. Auch wenn dies theoretisch der Fall sein sollte, kann es bei dem Versuchsaufbau  durch äußere Lichtquellen, somit Störungen, zu Abweichungen kommen. Auch kann eine leichte Schräglage des Spaltes das Licht zu beiden Seiten unterschiedlich stark brechen und somit zu abweichenden Ergebnissen führen. 

Auch ist auffällig, dass die Messwerte in den Minima nie null annehmen, trotz Abzug des Dunkelstromes $I_\text{d}$. Dies kann auf den ausgedehnten Messspalt zurückzuführen sein. Wenn die Minima schmaler als der Messspalt der Photodiode sind, kann das Minima nicht als solches aufgenommen werden, da auch Daten aus den aufsteigenden Flanken aufgenommen werden. 
Auch kann es durch das Erhitzen der Messapparaturen zu abweichenden Messergebnissen kommen.