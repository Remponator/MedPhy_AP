\section{Aufbau \& Durchführung}

Der Aufbau des Experiments besteht aus einem Laser, einem Spalt, einem Photelement, das auf einem Verschiebereiter angebracht ist, und einem Amperemeter.
Außerdem wird der Abstand $L$, welcher den Abstand vom Spalt zum Verschiebereiter beschreibt, gemessen.
Für möglichst genaue Ergebnisse wird ein Dunkelstrom $I_{du}$ gemessen, indem die Photodiode abgedeckt und der Strom am Amperemeter abgelesen wird.

\begin{figure}[H]
    \centering
    \includegraphics[height=5cm]{Aufbau/Aufbau.pdf}
    \caption{Aufbau des Experiments mit Laser,Spalt,Photoelement und Amperemeter. \cite{1}}
    \label{fig:Aufbau}
\end{figure}

Bei der Messung wird ein Einzelspalt in die Apparatur eingespannt.
Der Laser wird dabei auf den Spalt gerichtet, sodass auf der Photodiode das Hauptmaximum des Interferenzmusters liegt.
Danach wird das Photoelement über den Verschiebereiter in kleinen Abständen nach rechts und links verschoben.
Dabei wird sowohl der Strom $I$ als auch die verschobene Strecke $x$ aufgezeichnet.
Es sollten mindestens 50 Messwerte aufgezeichnet werden.
Analog dazu wird danach ein Einzelspalt mit einer anderen Spaltbreite $b$ eingespannt und die Messung wiederholt.
Im zweiten Teil des Experiments wird die Messung mit einem Doppelspalt analog wiederholt. \\
Anschließend werden über die gemessen Werte die Spaltbreiten der Einzelspalte und des Doppelspalts bestimmt.
