\documentclass{scrartcl}

\usepackage[aux]{rerunfilecheck}

\usepackage{fontspec}

\usepackage{amsmath}
\usepackage{amssymb}
\usepackage{mathtools}

\usepackage[
	math-style=ISO,
	bold-style=ISO,
	sans-style=italic,
	nabla=upright,
	partial=upright,
]{unicode-math}
\setmathfont{Latin Modern Math}


\usepackage{polyglossia}
\setmainlanguage{german}

%mehr Pakete hier

\usepackage[unicode]{hyperref}
\usepackage{bookmark}
%Einstellung hier, z.B. Fonts

\begin{document}
\section{Aufbau}

Bei diesem Versuch wird eine Drillachse verwendet um die Trägheitsmomente der verschiedenen Körper zu bestimmen. Die Achse ist durch eine Spiralfeder am Rahmen befestigt. Auf der Achse können die Körper befestigt und ausgelenkt werden. Der Auslenkwinkel kann auf einer Platte unter dem Körper abgelesen werden. Zu beachten ist hierbei, dass die Spiralfeder nicht mehr als 360° ausgelenkt werden darf, da es sonst zu plastischer Verformung kommen kann.

\section{Durchführung}

Zu Beginn müssen die Winkelrichtgröße $D$ und das Eigenträgheitsmoment $I_D$ der Apparatur bestimmt werden. 
Für die Bestimmung von $D$ wird eine (hier als masselos betrachtete) Stange verwendet, die an der Achse befestigt wird. Die Stange wird um einen Winkel $\rho$  ausgelenkt. Dabei misst man anhand einer Federwaage, die an der Stange befestigt wird, die Federkraft $F$. Die Federwage steht dabei senkrecht zur Stange (da sonst noch mit $sin(a)$ der Winkel der Waage berücksichtigt werden muss) und der Abstand zur Achse muss bei allen 10 Messung gleich gewählt sein. Die Winkelrichtgröße wird nun berechnet mit der Federkraft $F$, dem Abstand $r$ der Federwaage von der Achse und dem Auslenkwinkel \rho:

\begin{equation}
	D=\frac{F*r}{\rho}
\end{equation}

Nun werden an beiden Enden der Stange zwei Massen, dessen Gewicht und Maße zuvor bestimmt werden, befestigt. Durch Auslenken der Stange führt das System eine Schwingung durch, bei der die Periodendauer T gemessen wird. Diese Messung wird für 10 verschiedene Abstände wiederholt, wobei der Abstand der beiden Gewichte zur Achse identisch sein muss. Das Eigenträgheitsmoment $I_D$ wird nun anhand der Formeln (1), (2) und (3) berechnet.

Nachdem die Konstanten der Apparatur bestimmt sind, werden nun zwei verschiedene geometrische Körper auf die Achse gespannt. Die Körper werden aus ihrer Ruhelage ausgelenkt und erneut die Periodendauer T gemessen. Der Auslenkwinkel $\rho$ muss dabei notiert werden. Das Trägheitsmoment der Körper kann dann bestimmt werden. Die Messungen werden jeweils 5 mal wiederholt.

Zuletzt wird das Trägheitsmoment einer Holzpuppe in zwei verschiedenen Körperhaltungen bestimmt (siehe Skizze). Wie zuvor wird die Puppe ausgelenkt und die Periodenauer, sowie der gleichbleibende Auslenkwinkel $\rho$ gemessen. Das Trägheitsmoment der Puppe wird bestimmt, indem die Puppe vereinfacht als Zusammensetzung einzelner Zylinder betrachtet wird.

\end{document}