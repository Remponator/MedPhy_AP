\section{Diskussion}

In der Auswertung ist eine recht hohe Fehlerrate der Messwerte zu erkennen. Wird der Versuchsaufbau betrachtet, können diese Fehler auf statistische Zerfälle des Strahlers zurückzuführen sein. \*
Die Steigung des Plateaus ist mit 1\% pro 100V sehr flach, was auf ein sehr effektives Zählrohr schließen lässt. \\
Die Auswertung der Totzeit $T_\text{t}$ am Oszilloskop ist äußerst ungenau, da es zu starkem "Flackern" kommt und das Ablesen somit erschwert ist. Es ergaben sich folgende Werte:
\begin{gather*}
	 T_\text{t} =  58\pm1 \\
	T_\text{E} = 120\pm1
\end{gather*}
Die Auswertung durch die Zwei-Quellen-Methode funktioniert eher, hat jedoch auch eine große Fehleranfälligkeit. Hierbei ergibt sich eine Totzeit von $T_\text{t} = \SI{19\pm6}{\second}$. Somit weichen die beiden unterschiedlich gemessenen Totzeiten stark voneinander ab. Es ist jedoch anzunehmen, dass die durch die Zwei-Quellen-Methode gemessene Totzeit eher dem echten Wert entspricht.
Eine weitere Fehlerquelle kann das Amperemeter sein. Es ist möglich, dass das Amperemeter nicht empfindlich genug für die Messungen ist und somit die letzte Messung ungeeignet für eine genaue Messung der Ladung ist.