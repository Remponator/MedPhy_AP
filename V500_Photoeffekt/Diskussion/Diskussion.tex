\section{Diskussion}
Bei dem Versuch entstanden traten nur geringe Fehler. Es kann davon ausgegangen werden, dass trotz vieler möglicher Fehlerquellen, die Messdaten nahe an der Theorie liegen. Dies erkennt man an den Plots, beziehungsweise an den Regressionen, da diese sehr nah an den Werten liegen und zudem nur kleine Fehler liefern. 
Mögliche Fehlerquellen sind das Ablesen des Amperemeters, ein nicht genauer Aufbau der Apparaturen und eine Schwankung der Lichtintensität. Diese Schwankung kann die Messung bei der gelben Spektrallinie erklären. Dies ist die einzige Messreihe, welche ungenaue und teils nicht annehmbare Werte hervorbrachte, da der gemessene Photostrom immer wieder plötzlich abfiel. \*
Auch können externe Lichtquellen zu Abweichungen der Messwerte führen. \\
Es kann der Wert von h/$e_0$ mit dem Literaturwert verglichen werden. Der Literaturwert beträgt
\begin{equation*}
	h/e_0 = \SI{4,316e-15}{eV} .
\end{equation*}
Der anhand der Messwerte ermittelte Wert beträgt
\begin{equation*}
	h/e_0 = \SI{3,792e-15}{eV} .
\end{equation*}
Es besteht also eine Abweichung von 13,8\%.