\section{Diskussion}
Die Anregungsenergien bzw. die Wellenlängen für Quecksilber betragen experimentell ermittelt und als Literaturwert
\begin{equation}
    \lambda_\text{exp} = \SI{237.151 \pm 3.440}{nm} \qquad \text{und} \qquad \lambda_\text{Lit} = \SI{253}{nm}. \notag
\end{equation}
Die daraus resultierende relative Abweichung beträgt $6.3 \%$. 
Im Folgenden sind der Literaturwert und der experimentell ermittelte Wert der Ionisierungsenergien aufgelistet.
\begin{equation}
    E_\text{Ion,Lit} = \SI{10.438}{eV}. \notag
\end{equation}
\begin{equation}
    E_\text{Ion} = \SI{16.011 \pm 0.2}{eV}. \notag
\end{equation}
Der dabei auftretende Fehler liegt bei $ 53.4\%$.
Dies folgt aus Fehlern der Messgeräte und des Ablesens bei den aufgezeichneten Messkurven.
Des Weiteren lässt sich sagen, dass das Kontaktpotential aus einer ungenauen Messkurve entnommen wurde.
Dabei lassen sich Ablesefehler bei so kleinen Maßstäben kaum vermeiden.
Die unkonstante Temperatur und generell die Temperatureinstellungen haben ebenfalls Einfluss auf das Messergebnis gehabt.
Außerdem wurden zwei Kontaktpotentiale festgestellt, wobei Ersteres deutlich zu gro\ss{} ist und auch auf Messfehler schlie\ss{}en lässt.
\begin{equation}
  K_1 = \SI{10.561}{V} \qquad \text{und} \qquad K_2 = \SI{3.088}{V} \notag
\end{equation}
