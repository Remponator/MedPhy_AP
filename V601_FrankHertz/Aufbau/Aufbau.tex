\section{Aufbau und Durchführung}

Der grundlegende Aufbau des Frank-Hertz-Versuches wurde bereits in der Theorie \ref{sec:Theorie} beschrieben.
Zur Auswertung wird ein Milimeterpapier mit XY-Schreiber verwendet. 
\begin{figure}[H]
    \centering
    \includegraphics[height=4cm]{Aufbau/Aufbau.pdf}
    \caption{Aufbau des Frank-Hertz-Versuches.}
    \label{fig:Aufbau}
\end{figure}
Die Versuchsbestandteile werden wie in Abbildung \ref{fig:Aufbau} verbunden. 

Vor jeder Messung wird der XY-Schreiber passend eingestellt.
Im ersten Teil wird die integrale Energieverteilung der Elektronen aufgenommen. Hierzu wird bei Raumtemperatur (T $\approx$ 20°C) und bei einer Temperatur von 140 - 160°C der Auffängerstrom $I_\text{A}$ in Abhängigkeit von der Bremsspannung $U_\text{A}$ gemessen. Die Beschleunigungsspannung $U_\text{B}$ bleibt dabei bei einem konstanten Wert von 11V. Zum Erhitzen des Gases wird eine Kathodenheizung verwendet.\\
Im zweiten Teil wird die Ionisierungsspannung von Quecksilber bestimmt. Die Bremsspannung $U_\text{A}$ wird auf -30 V gestellt. Bei 100 - 110°C wird nun $I_\text{A}$ gegen $U_\text{B}$ aufgenommen.  \\
Zum Schluss werden drei Frank-Hertz-Kurven bei 160 - 200°C aufgenommen. $U_\text{B}$ wird dabei von 0 V bis 60 V verändert. 