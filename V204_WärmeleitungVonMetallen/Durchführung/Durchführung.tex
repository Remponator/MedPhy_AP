\section{Durchführung}
\subsection{Statische Methode}
Bei der statischen Methode wird an je zwei Stellen eines Stabes die Temperatur gemessen und der zeitliche Verlauf dargestellt.

Zu Beginn der Messungen muss überprüft werden, ob das Peltier Element richtig verkabelt ist, der Datenlogger angeschlossen ist und alle acht Thermoelement aufgenommen werden. Die Abtastrate beträgt dabei t = 5 s. Die Stäbe werden mit einer Isolierung bedeckt. 
Nun wird eine Spannung $U_\text{P}$ von 5V bei maximaler Stromstärke angelegt und bis zu einer Temperatur von 45 Grad Celsius am Thermoelement 7 gemessen. Danach werden die Proben wieder abgekühlt. 
Nach 540s werden die Temperaturen an den Thermoelementen T1, T4, T5 und T8 notiert.

\subsection{Dynamische Methode}
Bei der dynamischen Methode wird das Angsträm-Messverfahren angewandt. Hierbei werden die Proben periodisch erhitzt und die Wärmeleitfähigkeit aus der Ausbreitungsgeschwindigkeit der Temperaturwelle bestimmt.

Hierzu wird der Datenlogger auf eine Abtastrate von 2s eingestellt. Sobald die Thermoelemente eine Temperatur von 30 Grad oder kälter (nach der statischen Methode) erreicht haben, wird die Messung begonnen. Eine Spannung $U_\text{P}$ von 8V wird bei maximaler Stromstärke angelegt. Die Stäbe werden nun 40s lang erhitzt ('Heat'), dann 40s abgekühlt ('Cool') und dann wieder 40s lang erhitzt. Dieser Vorgang wird mehrmals wiederholt. 
Daraufhin werden die Probestäbe wieder hinuntergekühlt, damit zum Schluss eine Messung mit einer Periodendauer von 200s vorgenommen werden kann. Diese Messung wird beendet, sobald eine Temperatur von 80 Grad gemessen wird. 