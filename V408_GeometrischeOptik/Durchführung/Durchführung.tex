\newpage
\section{Durchführung}
\label{Durchführung}
In allen Versuchsteilen wird mithilfe der Halogenlampe, welche auf das Objekt, \\*die "Perl L", strahlt, ein Abbild auf den dahinter liegenden Schirm projiziert. Hierbei unterscheidet sich jedoch die Konstellation der Linsen.

\subsection{Versuchsteil A}
\label{TeilA}
Im ersten Versuchsteil wird bei fester Gegenstandsweite der Schirm so lange verschoben, bis ein scharfes Bild auf dem Schirm zu erkennen ist. Die Bildweite $b$ und Gegenstandsweite $g$ werden notiert. Daraufhin wird die Linse verschoben und der Versuch wiederholt. Insgesamt werden neun Messreihen aufgenommen.

\subsection{Versuchsteil B - Bessel}
\label{TeilB}
In diesem Versuchsteil wird zur Bestimmung der Brennweite $f$ die Methode von Bessel angewendet (siehe \ref{Theorie/Bessel}). Es wird vorgegangen wie beschrieben und die Weiten $g_1$, $g_2$, $b_1$ und $b_2$ notiert. Die Messung wird für neun verschiedene Abstände $e$ notiert.
Danach wird vor das Objekt ein zuerst blauer, dann roter Filter gesetzt und erneut jeweils fünf mal gemessen. 

\subsection{Versuchsteil C - Adde}
\label{TeilC}
In diesem Versuchsteil wird zur Bestimmung der Brennweite $f$ die Methode von Abbe angewendet (siehe \ref{Theorie/Abbe}). Die Sammell- und Zerstreuungslinse werden so nah aneinander geschoben, dass die Reiter sich berühren und ein konstanter Abstand gewährleistet werden kann. Die Linsen haben eine Brennweite $f$ +100 und -100. Nun wird vorgegangen wie bereits beschrieben. Der Abbildungsmaßstab $V$ und die Abstände $g'$ und $g$ notiert. Dies wird neun mal wiederholt.
