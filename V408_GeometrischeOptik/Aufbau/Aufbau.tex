\newpage
\section{Aufbau}
\label{Aufbau}
Der Versuch besteht aus einer optischen Bank, an deren einem Ende eine Halogenlampe befestigt ist. Die Halogenlampe leuchtet durch einen Gegenstand ("Perl L"), je nach Aufgabenteil verschiedene Linsen (Sammel-/Streuungslinse) auf einen am anderen Ende liegendem Schirm. \\*
In dem ersten Versuchsteilen wird der Aufbau aus Abbildung \ref{fig:Linsen}(a) verwendet. 
Im zweiten Versuchsteil wird der Aufbau aus Abbildung \ref{fig:Aufbau2} und für den letzten Versuchsteil der Aufbau aus Abbildung \ref{fig:Aufbau3} verwendet.
\begin{figure}[H]
  \centering
  \includegraphics[height=5cm]{Aufbau/Aufbau2.pdf}
  \caption{Aufbau des zweiten Versuchsteil - Methode von Bessel.}
  \label{fig:Aufbau2}
\end{figure}
\begin{figure}[H]
  \centering
  \includegraphics[height=5cm]{Aufbau/Aufbau3.pdf}
  \caption{Aufbau des zweiten Versuchsteil - Methode von Abbe.}
  \label{fig:Aufbau3}
\end{figure}
Im gesamten Versuch muss darauf geachtet werden, dass alle Versuchsbestandteile auf gleicher Höhe liegen.