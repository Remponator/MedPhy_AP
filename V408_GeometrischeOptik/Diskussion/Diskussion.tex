\section{Diskussion}
Die Auswertung des Experiments ergibt durch die verschiedenen Methoden unterschiedliche Ergebnisse.
Bei den ersten beiden Methoden wurde jeweils eine Sammellinse mit einer Herstellerangabe von $f_\text{Hersteller} = \SI{10}{cm}$ verwendet.
Während der Versuchsteil A bei der Messung eine Abweichung von $0.52 \%$ ergab, wurde am Graphen ein Wert mit einer Abweichung von $5.5 \%$ abgelesen.
Es kann also von einer Bestätigung des Abbildungsgesetzes (2) ausgegangen werden.
Die Methode von Bessel ergab für wei\ss{}es Licht eine Abweichung von $0.75 \%$.
Außerdem wurde bei der Bestrahlung mit rotem Licht eine um $1.2 \%$ größere Brennweite als bei blauem Licht berechnet.
Dies bestätigt die Aussage der chromatischen Abberation, dass kurzwelliges Licht aufgrund der Dispersion stärker gebrochen wird.
Im letzten Versuchsteil, der Methode von Abbe, wurde eine Zerstreuungslinse und eine Sammellinse hintereinander aufgestellt.
Es ergibt sich für die Brechkraft nach Gleichung (3) eine theoretische Brechkraft von $D 0 =$ und somit eine theoretisch unendliche Brennweite.
Die Messung ergibt jedoch eine endliche Brennweite, was auf Fehler der Linsen bzw. der Herstellerangaben hinwei\ss{}t.
Zusammenfassend lässt sich sagen, dass das Abbildungsgesetz und die chromatische Abberation nachgewiesen wurden.
Die Methode von Abbe konnte praktisch nicht auf ihre Gültigkeit nachgewiesen werden.
